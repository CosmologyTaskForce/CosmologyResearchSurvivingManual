\section{How to survive the calculations of Special Relativity}

\subsection{Important Relations}
\begin{quotation}
Metric in use
\begin{equation}\eta_{\mu\nu}=\left(\begin{matrix}
	-1 & 0 & 0 & 0\\
	0 & 1 & 0 & 0\\
	0 & 0 & 1 & 0\\
	0 & 0 & 0 & 1\\
\end{matrix}\right)\end{equation}
\end{quotation}




\section{Quantities and Operations}

\paragraph{d'Alembertian}
d'Alembert operator, or wave operator, is the Lapace operator in Minkowski space.
\footnote{Actually, there are more general definations for Lapacian, which includes this d'Alembertian of course.}
\begin{eqnarray}
\Box\equiv \partial_\mu\partial^\nu&=&\eta_{\mu\nu}\partial^\mu \partial^\nu
\end{eqnarray}

In the usual {t,x,y,z} natural orthonormal basis,
\begin{eqnarray}
 \Box&=&-\partial_t^2+\partial_x^2+\partial_y^2+\partial_z^2 \\
&=&-\partial_t^2+\Delta^2 \\
&=&-\partial_t^2+\nabla
\end{eqnarray}

\begin{quotation}
On wiki \footnote{wiki:D'Alembert\_operator}, they give some applications to it.
\paragraph{klein-Gordon equation} $(\Box+m^2)\phi=0$
\paragraph{wave equation for electromagnetic field in vacuum} For the electromagnetic four-potential $\Box A^\mu=0$\footnote{Gauge}
\paragraph{wave equation for small vibrations} $\Box_c u(t,x)=0\rightarrow u_{tt}-c^2 u_{xx}=0$
\end{quotation}




\section{Fields and Particles}

\subsection{Energy-Momentum Tensor for Particles}

\begin{equation}
S_p \equiv -m c \int \int \mathrm d s\mathrm d\tau \sqrt{-\dot x ^\mu g_{\mu\nu} \dot x^\nu} \delta^4(x^\mu - x^\mu (s))    ,
\end{equation}
in which $x^\mu(s)$ is the trajectory of the particle. Then the energy density $\rho$ corresponds to $m\delta^4(x^\mu- x^\mu(s))$.

The Largrange density
\begin{equation}
\mathcal L = -\int\mathrm ds mc \sqrt{-\dot x^\mu g_{\mu\nu}\dot x^\nu}\delta^4(x^\mu - x^\mu(s))
\end{equation}

Energy-momentum density is $\mathcal T^{\mu\nu} = \sqrt{-g}T^{\mu\nu}$ is
\begin{equation} 
\mathcal T^{\mu\nu} = -2 \frac{\partial \mathcal L}{\partial g_{\mu\nu}}
\end{equation}

Finally,
\begin{eqnarray}
\mathcal T^{\mu\nu} &=& \int \mathrm ds \frac{mc\dot x^\mu \dot x^\nu}{\sqrt{-\dot x^\mu g_{\mu\nu} \dot x^\nu}} \delta(t-t(s))\delta^3(\vec x - \vec x(t)) \\
&=& m\dot x^\mu \dot x^\nu \frac{\mathrm d s}{\mathrm d t} \delta^3(\vec x - \vec x(s(t)))
\end{eqnarray}






\section{Theorems}

\subsection{Killing Vector Related}

\newtheorem{theorem}{}[chapter]

\begin{theorem}[]
$\xi^a$ is Killing vector field, $T^a$ is the tangent vector of geodesic line. Then $T^a\nabla_a(T^b\xi_b)=0$, that is $T^b\xi_b$ is a constant on geodesics.
\end{theorem}




\section{Topics}
\subsection{Redshift}

In geometrical optics limit, the angular frequency $\omega$ of a photon with a 4-vector $K^a$, measured by a observer with a 4-velocity $Z^a$, is $\omega=-K_aZ^a$.


\section{What's in the begining}

In cosmology, the frame used most frequently is the cosmic microwave background radiation (CMB) stationary frame,  not earth frame. We sometimes say it is earth frame because the speed of earth relative to CMB is rather small compared to the galaxies' movement observed by we earth beings.



\section{Constants And Physical Quantities}

\begin{itemize}
\item
Deceleration parameter of today, 
\[q=-\left( \frac{a}{\dot a}\ddot a \right)=-\frac{\ddot a}{\dot a}\frac{1}{H(a)}\]
Deceleration parameter is tightly related to Friedmann equation.
\footnote{Firedmann equation can be written in terms of deceleration parameter,\[q=1/2(1+3w)(1+k/(aH)^2)\]}


\end{itemize}


\subsection{Cosmographic Parameters}
\begin{enumerate}[\bf\tiny{Co-Graphy-Par}-1]
\item
Recession speed
\begin{equation}
v = H_0 \cdot d
\end{equation}

\item
Hubble time
\begin{equation}
	t_H = \frac 1 H_0
\end{equation}

\item
Hubble distance
\begin{equation}
	D_H = \frac c H_0 = c \cdot t_H
\end{equation}

\item
Dimensionless density parameters

Matter
\begin{equation}
	\Omega_M = \frac{8\pi G \rho_M |_{a=1}}{3H_0^2}
\end{equation}

Lambda
\begin{equation}
	\Omega_\Lambda = \frac{\Lambda c^2}{3H_0^2}
\end{equation}

Curvature can be viewed as some energy, too.
From Friedmann equation we can sort out it. Or equivalently
\begin{equation}
	\Omega_k = 1- \Omega_M- \Omega_\Lambda
\end{equation}

\end{enumerate}




\subsubsection{Redshifts and Distances}

\begin{enumerate}[\bf\tiny{Co-Dis}-1]
\item
Redshift defined in observation
\begin{equation}
	z = \frac{\nu_e}{\nu_o} -1 = \frac{\lambda_o-\lambda_e}{\lambda_e}
\end{equation}

\item
In linear range, given a radial velocity, redshift can be written as
\begin{equation}
	z = \sqrt{\frac{1 + v/c}{1 - v/c}} - 1
\end{equation}

\item
Peculiar velocity
\begin{equation}
	v_{pec} = c \frac {z_{obs} - z_{cos} }{ 1+z }  ,
\end{equation}
in which,$z_{obs}$ is the observed redshift, while $z_{cos}$ is the cosmological redshift. Cosmological redshift is the Hubble flow "due solely to the expansion of the universe".
{\bf This definition is valid when $v_{pec}$ is much smaller than speed of light, i.e., $v_{pec}\ll c$.}

\item
Scale factor
\begin{equation}
	a|_{t_0} = (1+z) a|_{t_e}  \label{eq-co-dis-scale_factor}  .
\end{equation}
In cosmology, we often use another form which is derived from \label{eq-co-dis-scale_factor} and setting $a|_{t_0}= 1$
\begin{equation}
	a = \frac 1{1+z}     .
\end{equation}
This is the scale factor we used in line element.

\item
Relative redshift
\begin{equation}
	z_{12} = \frac{a|_{t_1}}{a|_{t_2}}-1 = \frac{1+z_2}{1+z_1}
\end{equation}

\item
Line-of-sight comoving distance
\begin{equation}
	D_c = D_H\int^z_0 \frac{1}{E(z')}\mathrm dz'
\end{equation}
in which $E(z) \equiv \sqrt{\Omega_M (1+z)^3 + \Omega_\Lambda + \Omega_k (1+z)^2}$    .

\item
Transverse comoving distance
\begin{equation}
D_M = \frac{D_H}{\sqrt{\left\vert\Omega_k\right\vert}} f(\sqrt{\left\vert \Omega_k \right\vert}\frac {D_C}{D_H})
\end{equation}
in which 
\begin{equation}
f(x) = \Big\{
\begin{array}{lll}
\sinh(x), & \Omega_k>0 \quad &\text{hyperbola}\\
x, & \Omega_k = 0 &\text{flat}\\
\sin(x), & \Omega_k<0 &\text{parabola}
\end{array} .
\end{equation}


\item
Angular diameter distance
\begin{equation}
	D_A = \frac {1}{1+z} D_M
\end{equation}

We just divide the transverse distances by $1+z$
The advantage of it is that it is not singular at $z\rightarrow\infty$

\item
Luminosity distance
\begin{equation}
	D_L\equiv \sqrt{\frac{L}{4\pi S}}
\end{equation}
in which $L$ is bolometric luminosity\footnote{Bolometric luminosity: The total energy radiated by an object at all wavelengths, usually given in joules per second.}, $S$ is the bolometric flux.

Related to other distances
\begin{equation}
	D_L = (1+z)D_M = (1+z)^2 D_A
\end{equation}

\item
Distance modules
\begin{equation}
	DM \equiv 5 \log{D_L/10\mathrm pc}
\end{equation}

In Mpc unit,
\begin{equation}
	DM \equiv 5 \log{D_L} + 25
\end{equation}

\item
Comoving volume
\begin{equation}
	\mathrm dV_C = D_H \frac{(1+z)^2D_\Lambda^2}{E(z)}\mathrm d\Omega \mathrm dz
\end{equation}

\item
Lookback time
\begin{equation}
	t_L = t_H \int_0^z \frac{\mathrm dz'}{(1+z')E(z')}
\end{equation}
Lookback time is "the difference between the age $t_o$ of the Universe now (at observation) and the age $t_e$ of the Universe at the time the photons were emmitted (according to the object). "

\item
Probability of intersecting objects
\begin{equation}
	\mathrm dP = n(z)\sigma(z)D_H\frac{(1+z)^2}{E(z)}\mathrm dz
\end{equation}

"Given a population of objects with comoving number density $n(z)$ (number per unit volume) and cross section $\sigma(z)$ (area), what is the incremental probability $\mathrm dP$ atha a line of sight will intersect one of the objects in redshift interval $\mathrm dz$ at redshift $z$?"\footnote{{\bf arXiv:astro-ph/9905116v4}}


\end{enumerate}

This part is mostly taken from {\bf arXiv:astro-ph/9905116v4}

Page 418 of Gravitation And Cosmology: Principles And Applications Of The General Theory Of Relativity written by Weinberg in 1972.



















\section{The Homogeneous and Isotropic Universe}

It is always pointed out that most thoeries are based on the Cosmological Principle.

I have stated this principle of in the chapter telling the perturbation theory in cosmology.

\begin{quotation}
Empedocles: `God is an infinite sphere whose center is everywhere and circumference nowhere.'

Cosmological principle states that VIEWED on a sufficiently large scale, the properties of the Universe are the same for all observers. The key to understand this is that "same" means same physics principles and physical constants. 
Wikipedia gives three qualifications and two testable consequences. The two consequences are isotropic and homogeneous. Isotropic indicates no matter where we are looking at the universe is the same while homogeneous indicates no matter where we are located the universe is about the same (i.e., we are looking at a fair sample of the universe).

Problem is how to describe isotropy and homogeneity. [Liang, P360]
\end{quotation}

Actually, we have a rigorous form for the principle, as I mentioned in these paragraphs.

\paragraph{Homogeneous} 
Space-time $(M,g_{ab})$, sliced into ${\Sigma_t}$, induced metric of $g_{ab}$ on $\Sigma_t$. $(M, g_{ab})$ is spatially homogeneous if we can always find ${\Sigma_t}$, ensuring the existance of a set of isometry on $\Sigma_t$ itself.








































\section{Quantities}

\subsection{Energy-momentum Tensor}

Energy-momentum tensor can be made clear using a standard procedure of field theory.

Check out the table below. [From Ohanian and Ruffini's GRAVATATION AND SPACETIME, P494.]

\begin{center}
\begin{tabular}{ccc}
\hline\hline
                 & Particle & Field   \\
\hline
\vspace{1ex}Quantities & $q_i(t)$    &   $\phi(\vec x,t)$ \\
\vspace{1ex}independent quantitie & $t$,i[?]    &  $\vec x,t$ \\
\vspace{2ex}Lagrange    & $L=L(q_i,\dot{q_i})$  & $L=\int \mathscr L(\phi,\partial \phi/\partial x^\mu)\mathrm d^3x$ \\
\vspace{2ex}
Equation of motion & $\frac{\mathrm d}{\mathrm d t}\frac{\partial L}{\partial \dot{q_i}}-\frac{\partial L}{\partial q_i}=0$ & $\frac{\partial}{\partial x^\mu}\frac{\partial\mathscr L}{\partial \phi_{,\mu}}-\frac{\partial \mathscr L}{\partial \phi}=0$\\
\vspace{2ex}
Hamiltonian & $H=\sum \dot{q_i}\frac{\partial L}{\partial \dot q}-L$  & $H=\int (\phi_{,0}\frac{\partial \mathscr L}{\partial \phi_{,0}}-\mathscr L)\mathrm d^3x$\\
\hline\hline
\end{tabular}
\end{center}




The integrand in $H$ is the energy density of the complete system. We would like to choose the $t_0^{\phantom00}$ component of a canonical energy-momentum tensor energy-momentum tensor to describe energy density,
\begin{equation}
t_0^{\phantom00}=\phi_{,o}\frac{\partial\mathscr L}{\partial \phi_{,0}}-\mathscr L
.\end{equation}

It can be proved that the following equation is the only one that satisfies the constrain we proposed before.
\begin{equation}
t_\mu^{\phantom\mu\nu}=\phi_{,\mu}\frac{\partial \mathscr L}{\partial \phi_{,\nu}}-\delta_\mu^\nu\mathscr L
\end{equation}

We have the conservation law: 
\begin{equation}
\frac{\partial}{\partial x^\nu}t_\mu^{\phantom\mu\nu}=0
\end{equation}

Start from this we can find the Hamiltonian is conserved.
\begin{equation}
\frac{\mathrm dH}{\mathrm d t}=0
\end{equation}

The total momentum is conserved too.
\begin{equation}
P_k=\int t_k^{\phantom k0}\mathrm d^3x
\end{equation}

This is all about a scalar field. If we switch to vector field (EM field) and tensor field (1-order gravitation field), the canonical energy-momentum tensors are
\begin{eqnarray}
{t_{(em)}}_{\mu}^{\phantom\mu\nu}&=&A_{,\mu}^\alpha\frac{\partial \mathscr L_{(em)}}{\partial A_{,\nu}^\alpha}-\delta_\mu^\nu\mathscr L_{(em)} \\
{t_{(g1)}}_\mu^{\phantom\mu\nu}&=&h_{,\mu}^{\alpha\beta}\frac{\partial \mathscr L_{(g1)}}{\partial h_{,\nu}^{\alpha\beta}}-\delta_\mu^\nu\mathscr L_{(g1)}
\end{eqnarray}







In special relativity, the energy-momentum tensor for ideal fluid is 
\begin{eqnarray}
T_{ab}=(\rho+p)U_aU_b+p\eta_{ab}
\end{eqnarray}

Each component has its physical meaning.
\begin{enumerate}
\item
$T^{00}$ is the energy density; 
\item
$T^{0k}=T^{k0}$ is the $k$ momentum density (energy flux density);
\item
$T^{kl}=T^{lk}$ is the $k$ momentum flux density in $l$ direction.
\end{enumerate}

Just change all the $\eta_{ab}$ into $g_{ab}$.

For any system,
\begin{eqnarray}
\nabla_\mu T^{\mu\nu}&=&U^\mu U^\nu\nabla_\mu \rho+U^\mu U^\nu\nabla_\mu p+(\rho+p)U^\mu\nabla_\mu U^\nu+(\rho+p)U^\nu\nabla_\mu U^\mu+g^{\mu\nu}\nabla_\mu p  \\
&=&U^\mu U^\nu \dot\rho+(g^{\mu\nu}+U^\mu U^\nu)\nabla_\mu p+(\rho+p)A^\nu+(\rho+p)U^\nu\nabla_\mu \Theta \\
&=&U^\mu U^\nu\dot\rho+(\rho+p)U^\nu\nabla_\mu \Theta \\
&=&Q^\nu
\end{eqnarray}

$A^\nu$ vanishes because we have chosen a comoving observer. $\nabla_\mu p$ vanishes because in our standard model $\rho,p$ are coordinate free.
Actually, since $U^\mu$ is timelike, we have $(g^{\mu\nu}+U^\mu U^\nu)\nabla_\mu p=h^{\mu\nu}\nabla_\mu p$.






\subsection{Friedmann Universe}

\paragraph{Cosmological Principle}
Empedocles: `God is an infinite sphere whose center is everywhere and circumference nowhere.'

Cosmological principle states that VIEWED on a sufficiently large scale, the properties of the Universe are the same for all observers. The key to understand this is that "same" means same physics principles and physical constants. 
Wikipedia gives three qualifications and two testable consequences. The two consequences are isotropic and homogeneous. Isotropic indicates no matter where we are looking at the universe is the same while homogeneous indicates no matter where we are located the universe is about the same (i.e., we are looking at a fair sample of the universe).

Problem is how to describe isotropy and homogeneity. [Liang, P360]

\paragraph{Robertson-Walker Metric}
The next problem is how to set up a simple and useful coordinate system which metric to use in cosmology. [Liang, P366]

\paragraph{Evolution of Scale Factor}
Friedmann equation
\begin{eqnarray}
3(\dot a^2+k)/a^2&=&8\pi p \label{Friedmann_eqn} \\
2\ddot a/a+(\dot a^2+k)/a^2&=&-8\pi p  \label{ij-components}
\end{eqnarray}

\ref{Friedmann_eqn} is called Friedmann equation.

It is better to use another set of equations which are identical to \ref{Friedmann_eqn} and \ref{ij-components}
\begin{eqnarray}
\ddot a&=&-4\pi a(\rho+3p)/3 \\
0&=&\dot\rho+3(\rho+p)\dot a/a
\end{eqnarray}

These equations can be solved in some circumstances. [Liang, P376]

It could be useful to reform these equation.

\begin{eqnarray}
H^2+\frac{k}{a^2}&=&\frac{8}{3}\pi \rho    \\
2\dot H + 3H^2 +\frac{k}{a^2}&=&-8\pi p
\end{eqnarray}

Mixing
\begin{eqnarray}
3\ddot a&=& -4\pi a(\rho +3p)  \\
0 &=& \dot \rho +3 (\rho +p )\frac{\dot a}{a}
\end{eqnarray}


\begin{equation}
\dot H = -4\pi G (\rho + p) + \frac{k}{a^2}
\end{equation}
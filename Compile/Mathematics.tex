\section{Differential Geometry}

\subsection{Metric}

\subsubsection{Definations}

Denote the basis in use as $\hat e_\mu$, then the metric can be written as
\begin{equation}
g_{\mu\nu}=\hat e_\mu \hat \cdot e_\nu
\end{equation}
if the basis satisfies

Inversed metric
\begin{equation}
g_{\mu\lambda}g^{\lambda\nu}=\delta_\mu^\nu = g_\mu^\nu
\end{equation}





\subsubsection{How to calculate the metric}

Let's check the definition of metric again.

If we choose a basis $\hat e_\mu$, then a vector (at one certain point) in this coordinate system is
\begin{equation}
x^a=x^\mu \hat e_\mu
\end{equation}

Then we can construct the expression of metric of this point under this coordinate system,
\begin{equation}
g_{\mu\nu}=\hat e_\mu\cdot \hat e_\nu
\end{equation}

For example, in spherical coordinate system, 
\begin{equation}
\vec x=r\sin \theta\cos\phi \hat e_x+r\sin\theta\sin\phi \hat e_y+r\cos\theta \hat e_z \label{eq:relativity_metric_point}
\end{equation}



Now we have to find the basis under spherical coordinate system. Assume the basis is $\hat e_r, \hat e_\theta, \hat e_\phi$. Choose some scale factors $h_r=1, h_\theta=r, h_\phi=r\sin\theta$. Then the basis is
$\hat e_r=\frac{\partial \vec x}{h_r\partial r}=\hat e_x \sin\theta\cos\phi+\hat e_y \sin\theta\sin\phi+\hat e_z \cos\theta$, etc. Then collect the terms in formula \ref{eq:relativity_metric_point} is we get $\vec x=r\hat e_r$, this is incomplete. So we check the derivative.
\begin{eqnarray}
\mathrm d\vec x&=& \hat e_x (\mathrm dr \sin\theta\cos\phi+r\cos\theta\cos\phi\mathrm d\theta-r\sin\theta\sin\phi\mathrm d\phi)\\
&&\hat e_y (\mathrm dr\sin\theta\sin\phi+r\cos\theta\sin\phi\mathrm d\theta+r\sin\theta\cos\phi\mathrm d\phi) \\
&&\hat e_z (\mathrm dr\cos\theta-r\sin\theta\mathrm d\theta) \\
&=&\mathrm dr(\hat e_x\sin\theta\cos\phi +\hat e_y \sin\theta\sin\phi -\hat e_z \cos\theta)  \\
&&\mathrm d\theta (\hat e_x\cos\theta\cos\phi +\hat e_y \cos\theta\sin\phi - \hat e_z \sin\theta)r \\
&&\mathrm d\phi (-\hat e_x\sin\phi +\hat e_y \cos\phi)r\sin\theta  \\
&=&\hat e_r\mathrm dr+\hat e_\theta r\mathrm d\theta +\hat e_\phi r\sin\theta\mathrm d \phi
\end{eqnarray}

Once we reach here, the component ($e_r ,e_\theta, e_\phi$) of the point under the spherical coordinates system basis ($\hat e_r, \hat e_\theta, \hat e_\phi$) at this point are clear, i.e.,

\begin{eqnarray}
\mathrm d\vec x&=&\hat e_r\mathrm d r+\hat e_\theta r\mathrm d \theta+\hat e_\phi r\sin\theta \mathrm d\phi \\
&=&e_r\mathrm d r+e_\theta \mathrm d\theta+e_\phi \mathrm d\phi
\end{eqnarray}

In this way, the metric tensor for spherical coordinates is 
\begin{equation}
g_{\mu\nu}=(e_\mu\cdot e_\nu)=\left(\begin{matrix}
1 &0&0 \\
0& r^2&0 \\
0&0& r^2\sin^2\theta \\
\end{matrix}\right)
\end{equation}



\subsection{Connection}

First class connection can be calcuated 
\begin{equation}
\Gamma^\mu_{\phantom{\mu}\nu\lambda}=\hat e^\mu\cdot \hat e_{\mu,\lambda}
\end{equation}

Second class connection is\footnote{Kevin E. Cahill}
\begin{equation}
[\mu\nu,\iota]=g_{\iota\mu}\Gamma^\mu_{\phantom{\mu}\nu\lambda}
\end{equation}




\subsection{Gradient, Curl, Divergence, etc}

\paragraph{Gradient} 
\begin{equation}
T^b_{\phantom bc;a}= \nabla_aT^b_{\phantom bc}=T^b_{\phantom bc,a}+\Gamma^b_{ad}T^d_{\phantom dc}-\Gamma^d_{ac}T^b_{\phantom bd}
\end{equation}

\paragraph{Curl}For an anti-symmetric tensor, $a_{\mu\nu}=-a_{\nu\mu}$
\begin{eqnarray}
\mathrm{Curl}_{\mu\nu\tau}(a_{\mu\nu})&\equiv& a_{\mu\nu;\tau}+a_{\nu\tau;\mu}+a_{\tau\mu;\nu} \\
&=&a_{\mu\nu,\tau}+a_{\nu\tau,\mu}+a_{\tau\mu,\nu}
\end{eqnarray}

\paragraph{Divergence}

\begin{eqnarray}
\mathrm{div}_\nu(a^{\mu\nu})&\equiv& a^{\mu\nu}_{\phantom{\mu\nu};\nu}=\frac{\partial a^{\mu\nu}}{\partial x^\nu}+\Gamma^\mu_{\nu\tau}a^{\tau\nu}+\Gamma^\nu_{\nu\tau}a^{\mu\tau} \\
&=&\frac1{\sqrt{-g}}\frac{\partial}{\partial x^\nu}(\sqrt{-g}a^{\mu\nu})+\Gamma^\mu_{\nu\lambda}a^{\nu\lambda}
\end{eqnarray}

For an anti-symmetric tensor
\begin{equation}
\mathrm {div}(a^{\mu\nu})=\frac1{\sqrt{-g}}\frac{\partial}{\partial x^\nu}(\sqrt{-g}a^{\mu\nu})
\end{equation}

\subparagraph{Annotation} Using the relation $g=g_{\mu\nu}A_{\mu\nu}$, $A_{\mu\nu}$ is the algebraic complement, we can prove the following two equalities.
\begin{equation}
\Gamma^\mu_{\mu\nu}=\partial_\nu\ln{\sqrt{-g}}
\end{equation}

\begin{equation}
V^\mu_{\phantom\mu;\mu}=\frac1{\sqrt{-g}}\frac{\partial}{\partial x^\mu}(\sqrt{-g}V^\mu)
\end{equation}

In some simple case, all the three kind of operation can be demonstrated by different applications of the del operator, which $\nabla\equiv \hat x\partial_x+\hat y\partial_y+\hat z \partial_z$. \\
Gradient,  $\nabla f$, in which $f$ is a scalar. \\
Divergence, $\nabla\cdot \vec v$ \\
Curl, $\nabla \times \vec v$
Laplacian, $\Delta\equiv \nabla\cdot\nabla\equiv \nabla^2$


\section{Linear Algebra}

\subsection{Basic Concepts}

\paragraph{Trace}
Trace should be calculated using the metrc. An example is the trace of Ricci tensor,
\begin{equation}
R=g^{ab}R_{ab}
\end{equation}

Einstein equation is \begin{equation}
R_{ab}-\frac{1}{2}g_{ab}R=8\pi G T_{ab}
\end{equation}
 The trace is \begin{eqnarray}
g^{ab}R_{ab}-\frac{1}{2}g^{ab}g_{ab}R&=&8\pi G g^{ab}T_{ab} \\
\Rightarrow R-\frac{1}{2} 4 R &=& 8\pi G T \\
\Rightarrow -R&=&8\pi GT
\end{eqnarray}






\section{Differential Equations}

\subsection{Standard Procedure}


\subsection{Tricky}

\paragraph{WKB Approximation}

When the highest derivative is multiplied by a small parameter, try this.
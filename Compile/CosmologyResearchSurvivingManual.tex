% !TEX program = xelatex

%This document had would had been used on Ways to Singularity, which is a website that supports MathJax. So some html elements may occur in this document. DELETE THIS when publishing. (If you are not very clear on the grammar I used here, read the academic publication called Time Traveller's Handbook of 1001 Tense Formations by Dr Dan Streetmentioner, which had would had been publish in year 220010 of Gregorian Calendar.)

\documentclass[12pt,a4paper]{book}

\usepackage{amsthm,amsfonts,amssymb,bm}
\usepackage[fleqn]{amsmath}

% Page settings
\addtolength{\textheight}{2.0cm}
\addtolength{\voffset}{-2cm}
\addtolength{\hoffset}{-1.0cm}
\addtolength{\textwidth}{2.0cm}

%\allowdisplaybreaks

%\usepackage{subeqnarray}
\usepackage{mathrsfs}
%\usepackage{color}
%\usepackage{url}
%\usepackage{ulem}
\usepackage{indentfirst}   % Indent first line of a paragraph
%\usepackage{textcomp}

\usepackage{enumerate}


%%Here is the configuration for chinese. setmainfont is the default font of the text.
\usepackage[cm-default]{fontspec}
\usepackage{xunicode}
%\usepackage{xltxtra}
\setmainfont{Arial}
%\setsansfont[BoldFont=Arial]{KaiTi_GB2312}
%\setmonofont{NSimSun}




%\XeTeXlinebreaklocale "zh"
%\XeTeXlinebreakskip = 0pt plus 1pt

% Figure, Diagram, Caption settings
%\usepackage{tikz}
%\usetikzlibrary{mindmap,trees}
\usepackage{graphicx}
%\usepackage{graphics}
%\usepackage[hang,small,bf]{caption}
%\setlength{\captionmargin}{50pt}

% Redefine some fonts.
\newfontfamily\heiti{"黑体"}
\newfontfamily\fs{"仿宋"}
\newfontfamily\yahei{"微软雅黑"}


%includeonly{}


\graphicspath{{Figures/}{Figures/Chap1/}{Figures/Chap2/}{Figures/Chap3/}}


\begin{document}
\title{Research Survival Handbook \\ (\textbf{Unfinished})}
\author{{\bf MA} Lei  \\
@ Interplanetary Immigration Agency \\
{\small\em \copyright \ Draft date \today}}
\date{}
%\begin{document}
\maketitle

% Redefine some math commands and environments.

\newcommand{\dd}{\mathrm d}
%\newcommand{\HH}{\mathcal H}
%\newcommand{\CN}{{\it Cosmologia Notebook}}
\newenvironment{eqnset}
{\begin{equation}\left \bracevert \begin{array}{l}}
{\end{array} \right. \end{equation}}

\newenvironment{eqn}
{\begin{equation}\left \bracevert \begin{array}{l}}
{\end{array} \right. \end{equation}}




\frontmatter
\tableofcontents



%%%%Inclule thse files please.%%%

\chapter{Preface}

I have a bad memory that I can hardly remember anything.

To get rid of it, I tried many ways of help myself memorizing things and pushing myself to the frontier of physics. Finally, I decided to collect some of my notes together and established this project on github.\url{http://cosmologytaskforce.github.com/CosmologyResearchSurvivingManual}

This is only a draft handbook for myself in principal. However, I believe everyone need a handbook of his/her area and my version of handbook might be helpful for some people working on similiar things with mine.

This book can never not be delivered formally because I borrowed many resources in this book. Here I list the most important of them here.

\begin{itemize}
\item
In the part of physics, I just take the source file of a note on GRE Physics subject test written by Lin Cong and typeset by Duncan Watts and rearranged the sections.\url{http://www.hcs.harvard.edu/~physics/?q=node/13};\url{http://www.hcs.harvard.edu/~physics/files/GRE\%20notes.tex} Then I did modifications based on it.
He said in his notes everyone is welcome to typeset and improve his notes but if you are going to use this document commercially you need to contact him first.
\end{itemize}


\mainmatter
\part{Fundamental Physics}

\chapter{Basic}










\section{Dimension}

How to find the relationship between two quantities? For example, what is the dimensional relationship between length and mass.
\begin{quotation}
Plank constant: $\hbar \sim [Energy]\cdot [Time] \sim [Mass]\cdot [Length]^2 \cdot [Time]^{-1}$ 

Speed of light in vacuum: $c\sim [Length]\cdot [Time]^{-1}$

Gravitational constant: $G \sim [Length]^3\cdot [Mass]^{-1} \cdot [Time]^{-2}$
\end{quotation}

Then it is easy to find that a combination of $c/\hbar$ cancels the dimension of mass and leaves the inverse of length. That is
\begin{equation}
[Length]^2 = \frac{\hbar G}{c^3}
\end{equation}



\section{Most Wonderful Equations That Should Never Be Forgotten}

\subsection{Electrodynamics}

\subsubsection{Maxwell Equations}
\begin{eqnarray}
\nabla\times\vec E&=&-\partial_t \vec B \\
\nabla\times\vec H&=&\vec J+\partial_t \vec D \\
\nabla\cdot \vec D&=&\rho \\
\nabla\cdot \vec B&=&0
\end{eqnarray}

For linear meterials, \begin{eqnarray}
\vec D&=&\epsilon \vec E \\
\vec B&=&\mu \vec H \\
\vec J&=& \sigma \vec E
\end{eqnarray}


\subsection{Dynamics}

Hamilton conanical equations

\begin{eqnarray}
\dot q_i &=& \frac{\partial H}{\partial p_i}  \\
\dot p_i &=& - \frac{\partial H}{\partial q_i}
\end{eqnarray}


\subsection{Thermaldynamics and Statistical Physics}

Liouville's Law

\begin{eqnarray}
\frac{\mathrm d \rho}{\mathrm d t}\equiv \frac{\partial \rho}{\partial t} + \sum_i \left[ \frac{\partial \rho}{\partial q_i}\dot q_i + \frac{\partial \rho}{\partial p_i}\dot p_i \right] = 0
\end{eqnarray}































\chapter{Classical Mechanics}




\begin{itemize}
\item A worked example on velocity and acceleration in a curved path in
a a plane: (the idea is to skillfully use $d(AB)=AdB+BdA$. This applies
to change of momentum as well.)\[
\hat{r}=\hat{i}\cos\theta+\hat{j}\sin\theta,\qquad\hat{\theta}=-\hat{i}\sin\theta+\hat{j}\cos\theta\]
\[
\hat{v}=\frac{d(R\hat{r})}{dt}=\frac{dR}{dt}\hat{r}+R\frac{d\hat{r}}{dt}=\dot{R}\hat{r}+R\omega\hat{\theta}\]
Similarly, \[
\vec{a}=(\ddot{R}-R\omega^{2})\hat{r}+(R\ddot{\theta}+2\dot{R}\dot{\theta})\hat{\theta}\]

\item Firing rocket\[
(v_{g}-v)dM+d(MV)=0\]
$M$ is rocket mass, $v$ is speed, $v_{g}$ is relative speed of
the waste fired out.
\item Bernoulli's equation\[
P+\frac{1}{2}\rho v^{2}+\rho gy=\text{const}\]
(conservation of energy)
\item Torricelli's Theorem: The outlet speed is the free-fall speed. For
a barrel with water depth $d$, an outlet at base has horizontal flow
speed $v=\sqrt{2gd}$.
\item Stoke's law: viscous drag is $6\pi\eta r_{s}\nu$.
\item Poiseille's Law:\[
\Delta P=\frac{8\mu LQ}{\pi r^{4}}\]
where $L$ is length of tube, $Q$ is volume rate. This describes
viscous incompressible flow through a constant circular cross-section.
\item Kepler's laws.

\begin{itemize}
\item An orbiting body travels in an ellipse\[
r(\theta)=\frac{a(1-e^{2})}{1+e\cos\theta}\]

\item {}``A line joining a planet and the Sun sweeps out equal areas during
equal intervals of time.''\[
\frac{d}{dt}\left(\frac{1}{2}r^{2}\dot{\theta}\right)=0\]
or\[
\frac{dA}{dt}=\frac{1}{2}r^{2}\dot{\theta}=\text{constant}\]

\item {}``The square of the orbital period of a planet is directly proportional
to the cube of the semi-major axis of its orbit.''\[
P=\frac{A}{dA/dt}=2\sqrt{\frac{\mu}{R}}R^{3/2}\qquad P^{2}\propto R^{3}\]
or \[
\frac{P^{2}}{a^{3}}=\frac{4\pi^{2}}{MG}\]

\end{itemize}
\item Coriolis force:\[
\vec{F}=-2m(\vec{\omega}\times\vec{v})\]

\item Diffusion: Fick's law. The diffusion flux is given by\[
\vec{J}_{r}=-D\nabla_{n}\phi\]

\item Frequency of a pendulum of arbitrary shape:\[
\omega=\sqrt{\frac{mgL}{I}}\qquad T=2\pi\sqrt{\frac{I}{mgL}}\]
where $L$ is the distance between the axis of rotation and the center
of mass.
\item Hamiltonian formulation:\[
\mathcal{H}=\sum_{i}p_{i}\dot{q}_{i}-\mathcal{L},\qquad\dot{p}=-\frac{\partial\mathcal{H}}{\partial q},\qquad\dot{q}=\frac{\partial\mathcal{H}}{\partial p}\]

\item Circular orbits exist for almost all potentials. Stable non-circular
orbits can occur for the simple harmonic potential and the inverse
square law.
\item Orbit questions:\[
V_{\text{eff}}(r)=V(r)+\frac{L^{2}}{2mr^{2}}\]
For a gravitational potential, $V(r)\propto\frac{1}{r}$. The total
energy of an object\[
E=\frac{1}{2}mv^{2}+V_{\text{eff}}\]
$E<V_{\text{min}}$ gives a spiral orbit, $E=V_{\text{min}}$ gives
a circular orbit,, $V_{\text{min}}<E<0$ gives an ellipse, $E=0$
is a parabolic orbit, and $E>0$ has a hyperbolic orbit.
\item If we want to approximate the equation of motion as a small oscillation
about a point of equilibrium $V'(x_{0})=0$ we can Taylor expand to
get\[
V(x)=V(x_{0})+\frac{1}{2}V''(x_{0})(x-x_{0})^{2}\]
and then get the force \[
F=-\frac{dV}{dx}=-V''(x_{0})(x-x_{0})\]
so that we can approximate small oscillations has harmonic oscillations
with $k=V''(x_{0})$ and\[
\omega=\sqrt{\frac{V''(x_{0})}{m}}.\]

\end{itemize}



















\chapter{Electromagnetism}


\begin{itemize}
\item Resistance is defined in terms of resistivity as\[
R=\frac{\rho L}{A}\]

\item Faraday's laws of electrolysis

\begin{itemize}
\item The mass liberated $\propto$ charge passed through
\item Mass of different elements liberated $\propto$ atomic weight/valence\[
m=\frac{QA}{Fv}\]
where $v$ is valence, $A$ is atomic weight in kg/kmol, $F=9.65\times10^{7}$C/kmol
(Faraday's constant)
\end{itemize}
\item Parallel plate capacitor $C=\epsilon_{0}A/d$ or $\epsilon A/d$ for
a dielectric. For a spherical capacitor,\[
C=\frac{4\pi\epsilon_{0}ab}{a-b}\]

\item In charging a capacitor,\[
q=q_{0}(1-e^{-t/RC})\]
discharging\[
q=q_{0}e^{-t/RC}\]

\item Cyclotron/magnetic bending\[
r=\frac{mv}{qB}\]

\item Torque experienced by a planar coil of $N$ loops, with current $I$
in each loop.\[
\tau=NIAB\sin\theta\]
where $\theta$ is the angle between $B$ and line perpendicular to
coil plane:\[
\vec{\tau}=\vec{\mu}\times\vec{B}\]

\item $B$-field of a long wire\[
B=\frac{\mu_{0}I}{2\pi r}\]
Center of a ring wire\[
B=\frac{\mu_{0}I}{2r}\]
Long solenoid\[
B=\mu_{0}nI\]
where $n$ is the turn density.
\item Ampere's Law:\[
\oint\mathbf{B}\cdot d\boldsymbol{\ell}=\mu_{0}I_{\text{enc}}\]

\item Conductors do not transmit EM wave, thus $\vec{E}$ vector is reversed
upon reflecting, $B$ vector is increased by a factor of 2 (by solving
propogation of EM wave).
\item Magnetic fields in matter:\[
B=\mu H=\mu_{0}(H+M)=\mu_{0}(H+\chi_{m}H)\]
Diamagnetic$\leftrightarrow\chi_{m}$ very small and negative. Paramagnetic,
$\leftrightarrow\chi_{m}$ small and positive, inversely proportional
to the absolute temperature. Ferromagnetic $\leftrightarrow\chi_{m}$
positive, can be greater than 1. $M$ is no longer proportional to
$H$.
\item For solenoid and toroid, $H=nI$, $n$ is the number density.
\item Self inductance:\[
\mathcal{E}=-L\frac{di}{dt}\]
$L$ is in henries, $1H=1V\cdot S/A=1J/A^{2}=1\text{ web}/A$\[
N\Phi=LI\]
is the flux linkage. Inductance of solenoid:\[
L=\frac{\mu N^{2}A}{c}\]

\item Induced e.m.f\[
|\mathcal{E}_{s}|=N\left|\frac{d\Phi_{B}}{dt}\right|\]

\item Time constant for $R-L$ circuit $t=L/R$. For an $R-C$ constant
$t=RC$. For an $L-C$ circuit, $\omega_{0}=1/\sqrt{LC}$.
\item $X_{L}=2\pi fL$ is the inductive reactance. $X_{C}=1/2\pi fC$ is
the capacitive reactance. The impedance is given by\[
Z=\sqrt{R^{2}+(X_{L}-X_{C})^{2}}\qquad\text{series}\]
\[
\frac{1}{Z}=\left[\left(\frac{1}{R}\right)^{2}+\left(\frac{1}{X_{C}}-\frac{1}{X_{L}}\right)^{2}\right]^{1/2}\qquad\text{parallel}\]
Current is maximized at resonance $X_{L}=\omega L=X_{C}=1/\omega C$
(there will be a lot of questions on this)
\item Larmor formula for radiation\[
P=\frac{\mu_{0}q^{2}a^{2}}{6\pi c}\propto q^{2}a^{2}\]
where $a$ is the acceleration. Energy per unit area decreases as
distance increases (inverse square relation).
\item Mean drift speed:\[
\vec{v}=\frac{\vec{J}}{ne}\]
where $n$ is the number of atoms per volume, $J$ is current density
$I/A$.
\item Impedance of capacitor \[
Z=\frac{1}{i\omega C}\]
Impedance of inductor \[
Z=i\omega L\]

\item Magnetic field on axis of a circle of current\[
B=\frac{\mu_{0}I}{2}\frac{r^{2}}{(r^{2}+z^{2})^{3/2}}\]

\item Bremsstrahlung: electromagnetic radiation produced by the deceleration
of a charged particle.
\item For incident wave reflecting off a plane, just set up a boundary value
problem.\[
E_{1}^{\perp}-E_{2}^{\perp}=\sigma\qquad E_{1}^{\parallel}=E_{2}^{\parallel}\]
and remember the Poynting vector \[
\vec{S}\propto\vec{E}\times\vec{B}\]
points in the direction of propagation.\[
E_{0}+E_{0}^{\text{reflected}}=E_{0}^{\text{transmitted}}\]

\item Lenz's law: The idea is the system responds in a way to restore or
at least attempt to restore to the original state.
\item Impedance matching to maximize power transfer or to prevent terminal-end
reflection.\[
Z_{\text{rad}}=Z_{\text{source}}^{*}\]
\[
I(X_{g})+I(X_{L})=IR\]
Generator impedance:\[
R_{g}+jX_{g}\]
Local impedance:\[
R_{L}+jX_{L}\]
\[
Z=R+j(\omega L+1/\omega C)\]

\item Propagation vector $\vec{k}$\[
\vec{E}(\vec{r},t)=\vec{E}_{0}e^{i(\vec{k}\cdot\vec{r}-\omega t)}\]
\[
\vec{B}(\vec{r},t)=\frac{1}{c}|\vec{E}(\vec{r},t)|\]
\[
(\hat{k}\times\hat{n})=\frac{1}{c}\hat{k}\times\hat{E}\]

\item No electric field inside a constant potential enclosure implies constant
$V$ inside.
\item Hall effect\[
R_{H}=\frac{1}{(p-n)e}\]
can be used to test the nature of charge carrier. $p$ for positive,
$n$ for negative.
\item Lorentz force\[
\vec{F}=q(\vec{E}+\vec{v}\times\vec{B})\]

\item $\nabla\cdot(\nabla\times\vec{H})=0$, $\nabla\times(\nabla f)=0$
\item One usually has cycloid motion whenever the electric and magnetic
fields are perpendicular.
\item Faraday's law:\[
\mathcal{E}=\vec{E}\cdot d\vec{L}=-\frac{d\Phi}{dt}\]

\item Visible spectrum in meters: Radio $10^{3}$ (on the order of buildings);
Microwave $10^{-2}$; Infrared $10^{-5}$; visible 700-900 nm ($10^{-6}$);
UV $10^{-8}$(molecules); X-ray $10^{-10}$(atoms); gamma ray $10^{-12}$
(nuclei)
\item Displacement field\[
\vec{D}=\epsilon_{0}\vec{E}+\vec{P}=\epsilon_{0}\vec{E}+\epsilon_{0}\chi_{e}\vec{E}=\epsilon_{0}(1+\chi_{e})\vec{E}=\epsilon\vec{E}\]
Dielectric constant\[
\epsilon_{r}=1+\chi_{e}=\frac{\epsilon}{\epsilon_{0}}\]
\[
\sigma_{b}=\vec{P}\cdot\vec{n}\]
\[
\rho_{b}=-\vec{\nabla}\cdot\vec{P}\]
These are the bound charge densities. Also note\[
\nabla\times\vec{D}=\nabla\times\vec{P}\]
is not necessarily zero.
\item We have\[
\vec{B}=\begin{cases}
\begin{array}{c}
\mu_{0}nI\hat{z}\\
0\end{array} & \begin{array}{c}
\text{inside a solenoid}\\
\text{outside a solenoid}\end{array}\end{cases}\]
where $n$ is density per length.\[
\vec{B}=\begin{cases}
\begin{array}{c}
\frac{\mu_{0}nI}{2\pi s}\hat{\phi}\\
0\end{array} & \begin{array}{c}
\text{inside a toroid}\\
\text{outside}\end{array}\end{cases}\]

\item Force per unit length between two wires:\[
f=\frac{\mu_{0}}{2\pi}\frac{I_{1}I_{2}}{d}\]

\item $B=\frac{\mu_{0}I}{4\pi s}(\sin\theta_{2}-\sin\theta_{1})$ looks
like the magnetic field due to a segment of wire, where $\theta_{i}$
is the angle from the normal.
\item Mutual inductance of two loops\[
M_{21}=\frac{\mu_{0}}{4\pi}\oint\oint\frac{d\vec{l}_{1}\cdot d\vec{l}_{2}}{r_{ij}}\]

\item Radiation pressure\[
P=\frac{I}{c}=\frac{\langle S\rangle}{c}\cos\theta\]
It's twice that for a perfect reflector.
\item $\nabla\cdot\vec{D}=\rho_{f}$ $\nabla\times\vec{H}=\vec{J}_{f}+\frac{\partial\vec{D}}{\partial t}$,
$\nabla\cdot\vec{B}=0$, $\nabla\times\vec{E}=-\frac{\partial\vec{B}}{\partial t}$.
\item Boundary conditions:\[
\epsilon_{1}\vec{E}_{1}^{\perp}-\epsilon_{2}\vec{E}_{2}^{\perp}=\sigma_{f}\qquad\vec{B}_{1}^{\perp}-\vec{B}_{2}^{\perp}=0\]
\[
\vec{E}_{1}^{\parallel}-\vec{E}_{2}^{\parallel}=0,\qquad\mu_{1}\vec{B}_{1}^{\parallel}-\mu_{2}\vec{B}_{2}^{\parallel}=\vec{k}_{f}\times\hat{n}\]

\item Biot-Savart law:\[
\vec{B}(\vec{r})=\frac{\mu_{0}I}{4\pi}\int\frac{d\vec{l}\times\vec{r}}{|\vec{r}^{3}|}\]

\item $B$-field at a center of a ring\[
\vec{B}=\frac{\mu_{0}I}{2r}\]

\item $H=\frac{1}{\mu_{0}}B-M$, $J_{b}=\nabla\times\vec{M}$, $\vec{k}_{b}=\vec{M}\times\hat{n}$\[
\vec{B}=\mu\vec{H},\qquad\mu=\mu_{0}(1+\chi_{m})\]

\end{itemize}













\chapter{Optics and Wave Phenomena}
\begin{itemize}
\item Speed of propagation for waves

\begin{itemize}
\item Transverse on string, $v=\sqrt{T/\rho}$
\item Longitudinal in liquid, $v=\sqrt{B/\rho}$, $B$ is bulk modulus
\item Longitudinal in solid, $v=\sqrt{Y/\rho},$$Y$ is Young's modulus
\item Longitudinal in gases, $v=\sqrt{\gamma P/\rho}$
\end{itemize}
\item For open pipe, fundamental frequency is $v/2L$ where $v$ is the
speed of sound. For a closed pipe it is $(2n-1)\lambda/4=L$. The
idea is $\lambda f=v$.
\item Speed of sound in air is \[
v=\sqrt{\frac{\gamma kT}{m}}=\sqrt{\frac{\gamma RT}{M}}\propto\sqrt{T}\]
where $m$ is the mass of a molecule, and $M$ is the molar mass in
kg/mole.
\item Resonant frequency of a rectangular drum\[
f_{mn}=\frac{\nu}{2}\sqrt{\left(\frac{m}{L_{x}}\right)^{2}+\left(\frac{n}{L_{y}}\right)^{2}}\]

\item Doppler effect\[
f''=\frac{v}{v+v_{\text{source}}}f\]
$v$ is the velocity in the medium, $v_{\text{source}}$ is the source
velocity w.r.t. medium. In general,\[
\frac{f_{\text{listener}}}{v\pm v_{\text{lis}}}=\frac{f_{\text{source}}}{v\pm v_{\text{source}}}\]
The $\pm$ can be determined by examining if the frequency received
is higher or lower.
\item Lens optics:\[
\frac{1}{p}+\frac{1}{q}=\frac{1}{f}\]
Sign convention, real image has positive sign.
\item Lens maker's equation:\[
\frac{1}{f}\approx(n-1)\left(\frac{1}{R_{1}}-\frac{1}{R_{2}}\right)\]
If $R_{1}$ is positive, it's convex, negative, concave. If $R_{2}$
is positive, it's concave, if it's negative, it's convex.
\item Young's double slit:\[
d\sin\theta=m\lambda\qquad\text{maxima}\]
\[
yd=mD\lambda\qquad d\ll D,\ \theta\text{ small}\]
\[
d\sin\theta=(m+\tfrac{1}{2})\lambda\qquad\text{minima}\]

\item If we have a slab of material with thickness $t$ and refractive index
$n_{2}$, and the other medium is $n_{1}$.\[
\frac{2n_{2}t}{n_{1}\lambda_{1}}=m+\frac{1}{2}\qquad\text{max}\]
\[
\frac{2n_{2}t}{n_{1}\lambda_{1}}=m+1\qquad\text{min}\]

\item Conversely: if we have three layers of material, $n_{1},$ $n_{t}$,
and $n_{2}$ (top to bottom), then we have a couple of different situations
that would like to a maximum in intensity:\[
d=\frac{m\lambda}{2n_{t}}\qquad n_{1}>n_{t}>n_{2},\qquad n_{1}<n_{t}<n_{2}\]
\[
d=\frac{(m+\frac{1}{2})\lambda}{2n_{t}}\qquad n_{1}<n_{t}>n_{2},\qquad n_{1}>n_{t}<n_{2}\]
I think it's fair to assume that the minima occur when you replace
$m+\frac{1}{2}$ with $m$ and vice-versa.
\item Diffraction grating \[
d\sin\theta=m\lambda\]
If incident at angle $\theta_{i}$ \[
d(\sin\theta_{m}+\sin\theta_{i})=m\lambda\]
The overall result is an interference pattern modulated by single
slit diffraction envelope. Intensity of interference\[
I=I_{0}\frac{\sin^{2}(N\phi/2)}{\sin^{2}(\phi/2)}\qquad\phi=\frac{2\pi}{\lambda}d\sin\theta\]
Minima occurs at $N\phi/2=\pi,\ldots n\pi$ where $n/N\notin\mathbb{Z}$.
Maxima occurs at $\phi/2=0,\pi,2\pi,\ldots.$ Single-slit envelope,\[
I=I_{0}\frac{\sin^{2}(\phi'/2)}{(\phi'/2)^{2}}\qquad\phi'=\frac{2\pi}{\lambda}w\sin\theta\]
where $w$ is the width of the slit. Overall,\[
I=I_{0}\frac{\sin^{2}(\phi'/2)\sin^{2}(N\phi/2)}{(\phi'/2)^{2}\sin^{2}(\phi/2)}\]

\item Bragg's law of reflection\[
m\lambda=2d\sin\theta\]
Make sure that $\theta$ is a glancing angle, not angle of incidence
(relative to the plane). This gives the angles for coherent and incoherent
scattering from a crystal lattice.
\item Index of refraction is defined as \[
n=\frac{c}{v}\]
Again,\[
n_{1}\sin\theta_{1}=n_{2}\sin\theta_{2}\]

\item Brewster's angle is the angle of incidence at which light with a particular
polarization is perfectly transmitted, no reflection.\[
\tan\theta=\frac{n_{2}}{n_{1}}\]

\item Diffraction again (more background info). The light diffracted by
a grating is found by summing the light diffracted from each of the
elements, and is essentially a convolution of diffraction and interference
pattern. Fresnel diffraction is near field, and fraunhofer diffraction
is far field.
\item Diffraction limited imaging\[
d=1.22\lambda N\]
where $N$ is the focal length/diameter. Angular resolution is \[
\sin\theta=1.22\frac{\lambda}{D}\]
where $D$ is the lens aperture.
\item Thin-film theory. Say the film has higher refractive index. Then there's
a phase change for reflection off front surface, no phase change for
reflection off back surface. Constructive interference thickness $t$:
$2t=(n+1/2)\lambda$. Destructive interference $2t=n\lambda$.
\item The key idea for many questions is to scrutinize path difference (optical)
\item Some telescopes have two convex lenses, the objective and the eyepiece.
For the telescope to work the lenses have to be at a distance equal
to the sum of their focal lengths, i.e. $d=f_{\text{objective}}+f_{\text{eye}}$:\[
M=\left|\frac{f_{\text{objective}}}{f_{\text{eye}}}\right|\]
Magnifying power = max angular magnification = image size with lens/image
size without lens.
\item Microscopy\[
\text{magnifying power}=\frac{\beta}{\alpha}\]

\item In Michelson interferometer a change of distance $\lambda/2$ of the
optical path between the mirrors generally results in a change of
$\lambda$ of optical path of light ray, thus potentially giving a
cycle of bright$\to$dark$\to$bright fringes.
\item Mirror with curvature $f\approx R/2$.
\item Beats: the beat frequency is $f_{1}-f_{2}$:\[
\sin(2\pi f_{1}t)+\sin(2\pi f_{2}t)=2\cos\left(2\pi\frac{f_{1}-f_{2}}{2}t\right)\sin\left(2\pi\frac{f_{1}+f_{2}}{2}t\right)\]

\end{itemize}








\part{Advanced Physics}

\chapter{Relativy}
\include{Relativy}

\part{Tools}

\chapter{Mathematics}
\section{Differential Geometry}

\subsection{Metric}

\subsubsection{Definations}

Denote the basis in use as $\hat e_\mu$, then the metric can be written as
\begin{equation}
g_{\mu\nu}=\hat e_\mu \hat \cdot e_\nu
\end{equation}
if the basis satisfies

Inversed metric
\begin{equation}
g_{\mu\lambda}g^{\lambda\nu}=\delta_\mu^\nu = g_\mu^\nu
\end{equation}





\subsubsection{How to calculate the metric}

Let's check the definition of metric again.

If we choose a basis $\hat e_\mu$, then a vector (at one certain point) in this coordinate system is
\begin{equation}
x^a=x^\mu \hat e_\mu
\end{equation}

Then we can construct the expression of metric of this point under this coordinate system,
\begin{equation}
g_{\mu\nu}=\hat e_\mu\cdot \hat e_\nu
\end{equation}

For example, in spherical coordinate system, 
\begin{equation}
\vec x=r\sin \theta\cos\phi \hat e_x+r\sin\theta\sin\phi \hat e_y+r\cos\theta \hat e_z \label{eq:relativity_metric_point}
\end{equation}



Now we have to find the basis under spherical coordinate system. Assume the basis is $\hat e_r, \hat e_\theta, \hat e_\phi$. Choose some scale factors $h_r=1, h_\theta=r, h_\phi=r\sin\theta$. Then the basis is
$\hat e_r=\frac{\partial \vec x}{h_r\partial r}=\hat e_x \sin\theta\cos\phi+\hat e_y \sin\theta\sin\phi+\hat e_z \cos\theta$, etc. Then collect the terms in formula \ref{eq:relativity_metric_point} is we get $\vec x=r\hat e_r$, this is incomplete. So we check the derivative.
\begin{eqnarray}
\mathrm d\vec x&=& \hat e_x (\mathrm dr \sin\theta\cos\phi+r\cos\theta\cos\phi\mathrm d\theta-r\sin\theta\sin\phi\mathrm d\phi)\\
&&\hat e_y (\mathrm dr\sin\theta\sin\phi+r\cos\theta\sin\phi\mathrm d\theta+r\sin\theta\cos\phi\mathrm d\phi) \\
&&\hat e_z (\mathrm dr\cos\theta-r\sin\theta\mathrm d\theta) \\
&=&\mathrm dr(\hat e_x\sin\theta\cos\phi +\hat e_y \sin\theta\sin\phi -\hat e_z \cos\theta)  \\
&&\mathrm d\theta (\hat e_x\cos\theta\cos\phi +\hat e_y \cos\theta\sin\phi - \hat e_z \sin\theta)r \\
&&\mathrm d\phi (-\hat e_x\sin\phi +\hat e_y \cos\phi)r\sin\theta  \\
&=&\hat e_r\mathrm dr+\hat e_\theta r\mathrm d\theta +\hat e_\phi r\sin\theta\mathrm d \phi
\end{eqnarray}

Once we reach here, the component ($e_r ,e_\theta, e_\phi$) of the point under the spherical coordinates system basis ($\hat e_r, \hat e_\theta, \hat e_\phi$) at this point are clear, i.e.,

\begin{eqnarray}
\mathrm d\vec x&=&\hat e_r\mathrm d r+\hat e_\theta r\mathrm d \theta+\hat e_\phi r\sin\theta \mathrm d\phi \\
&=&e_r\mathrm d r+e_\theta \mathrm d\theta+e_\phi \mathrm d\phi
\end{eqnarray}

In this way, the metric tensor for spherical coordinates is 
\begin{equation}
g_{\mu\nu}=(e_\mu\cdot e_\nu)=\left(\begin{matrix}
1 &0&0 \\
0& r^2&0 \\
0&0& r^2\sin^2\theta \\
\end{matrix}\right)
\end{equation}



\subsection{Connection}

First class connection can be calcuated 
\begin{equation}
\Gamma^\mu_{\phantom{\mu}\nu\lambda}=\hat e^\mu\cdot \hat e_{\mu,\lambda}
\end{equation}

Second class connection is\footnote{Kevin E. Cahill}
\begin{equation}
[\mu\nu,\iota]=g_{\iota\mu}\Gamma^\mu_{\phantom{\mu}\nu\lambda}
\end{equation}




\subsection{Gradient, Curl, Divergence, etc}

\paragraph{Gradient} 
\begin{equation}
T^b_{\phantom bc;a}= \nabla_aT^b_{\phantom bc}=T^b_{\phantom bc,a}+\Gamma^b_{ad}T^d_{\phantom dc}-\Gamma^d_{ac}T^b_{\phantom bd}
\end{equation}

\paragraph{Curl}For an anti-symmetric tensor, $a_{\mu\nu}=-a_{\nu\mu}$
\begin{eqnarray}
\mathrm{Curl}_{\mu\nu\tau}(a_{\mu\nu})&\equiv& a_{\mu\nu;\tau}+a_{\nu\tau;\mu}+a_{\tau\mu;\nu} \\
&=&a_{\mu\nu,\tau}+a_{\nu\tau,\mu}+a_{\tau\mu,\nu}
\end{eqnarray}

\paragraph{Divergence}

\begin{eqnarray}
\mathrm{div}_\nu(a^{\mu\nu})&\equiv& a^{\mu\nu}_{\phantom{\mu\nu};\nu}=\frac{\partial a^{\mu\nu}}{\partial x^\nu}+\Gamma^\mu_{\nu\tau}a^{\tau\nu}+\Gamma^\nu_{\nu\tau}a^{\mu\tau} \\
&=&\frac1{\sqrt{-g}}\frac{\partial}{\partial x^\nu}(\sqrt{-g}a^{\mu\nu})+\Gamma^\mu_{\nu\lambda}a^{\nu\lambda}
\end{eqnarray}

For an anti-symmetric tensor
\begin{equation}
\mathrm {div}(a^{\mu\nu})=\frac1{\sqrt{-g}}\frac{\partial}{\partial x^\nu}(\sqrt{-g}a^{\mu\nu})
\end{equation}

\subparagraph{Annotation} Using the relation $g=g_{\mu\nu}A_{\mu\nu}$, $A_{\mu\nu}$ is the algebraic complement, we can prove the following two equalities.
\begin{equation}
\Gamma^\mu_{\mu\nu}=\partial_\nu\ln{\sqrt{-g}}
\end{equation}

\begin{equation}
V^\mu_{\phantom\mu;\mu}=\frac1{\sqrt{-g}}\frac{\partial}{\partial x^\mu}(\sqrt{-g}V^\mu)
\end{equation}

In some simple case, all the three kind of operation can be demonstrated by different applications of the del operator, which $\nabla\equiv \hat x\partial_x+\hat y\partial_y+\hat z \partial_z$. \\
Gradient,  $\nabla f$, in which $f$ is a scalar. \\
Divergence, $\nabla\cdot \vec v$ \\
Curl, $\nabla \times \vec v$
Laplacian, $\Delta\equiv \nabla\cdot\nabla\equiv \nabla^2$


\section{Linear Algebra}

\subsection{Basic Concepts}

\paragraph{Trace}
Trace should be calculated using the metrc. An example is the trace of Ricci tensor,
\begin{equation}
R=g^{ab}R_{ab}
\end{equation}

Einstein equation is \begin{equation}
R_{ab}-\frac{1}{2}g_{ab}R=8\pi G T_{ab}
\end{equation}
 The trace is \begin{eqnarray}
g^{ab}R_{ab}-\frac{1}{2}g^{ab}g_{ab}R&=&8\pi G g^{ab}T_{ab} \\
\Rightarrow R-\frac{1}{2} 4 R &=& 8\pi G T \\
\Rightarrow -R&=&8\pi GT
\end{eqnarray}






\section{Differential Equations}

\subsection{Standard Procedure}


\subsection{Tricky}

\paragraph{WKB Approximation}

When the highest derivative is multiplied by a small parameter, try this.
\chapter{Cosmology}
\section{What's in the begining}

In cosmology, the frame used most frequently is the cosmic microwave background radiation (CMB) stationary frame,  not earth frame. We sometimes say it is earth frame because the speed of earth relative to CMB is rather small compared to the galaxies' movement observed by we earth beings.



\section{Constants And Physical Quantities}

\begin{itemize}
\item
Deceleration parameter of today, 
\[q=-\left( \frac{a}{\dot a}\ddot a \right)=-\frac{\ddot a}{\dot a}\frac{1}{H(a)}\]
Deceleration parameter is tightly related to Friedmann equation.
\footnote{Firedmann equation can be written in terms of deceleration parameter,\[q=1/2(1+3w)(1+k/(aH)^2)\]}


\end{itemize}


\subsection{Cosmographic Parameters}
\begin{enumerate}[\bf\tiny{Co-Graphy-Par}-1]
\item
Recession speed
\begin{equation}
v = H_0 \cdot d
\end{equation}

\item
Hubble time
\begin{equation}
	t_H = \frac 1 H_0
\end{equation}

\item
Hubble distance
\begin{equation}
	D_H = \frac c H_0 = c \cdot t_H
\end{equation}

\item
Dimensionless density parameters

Matter
\begin{equation}
	\Omega_M = \frac{8\pi G \rho_M |_{a=1}}{3H_0^2}
\end{equation}

Lambda
\begin{equation}
	\Omega_\Lambda = \frac{\Lambda c^2}{3H_0^2}
\end{equation}

Curvature can be viewed as some energy, too.
From Friedmann equation we can sort out it. Or equivalently
\begin{equation}
	\Omega_k = 1- \Omega_M- \Omega_\Lambda
\end{equation}

\end{enumerate}




\subsubsection{Redshifts and Distances}

\begin{enumerate}[\bf\tiny{Co-Dis}-1]
\item
Redshift defined in observation
\begin{equation}
	z = \frac{\nu_e}{\nu_o} -1 = \frac{\lambda_o-\lambda_e}{\lambda_e}
\end{equation}

\item
In linear range, given a radial velocity, redshift can be written as
\begin{equation}
	z = \sqrt{\frac{1 + v/c}{1 - v/c}} - 1
\end{equation}

\item
Peculiar velocity
\begin{equation}
	v_{pec} = c \frac {z_{obs} - z_{cos} }{ 1+z }  ,
\end{equation}
in which,$z_{obs}$ is the observed redshift, while $z_{cos}$ is the cosmological redshift. Cosmological redshift is the Hubble flow "due solely to the expansion of the universe".
{\bf This definition is valid when $v_{pec}$ is much smaller than speed of light, i.e., $v_{pec}\ll c$.}

\item
Scale factor
\begin{equation}
	a|_{t_0} = (1+z) a|_{t_e}  \label{eq-co-dis-scale_factor}  .
\end{equation}
In cosmology, we often use another form which is derived from \label{eq-co-dis-scale_factor} and setting $a|_{t_0}= 1$
\begin{equation}
	a = \frac 1{1+z}     .
\end{equation}
This is the scale factor we used in line element.

\item
Relative redshift
\begin{equation}
	z_{12} = \frac{a|_{t_1}}{a|_{t_2}}-1 = \frac{1+z_2}{1+z_1}
\end{equation}

\item
Line-of-sight comoving distance
\begin{equation}
	D_c = D_H\int^z_0 \frac{1}{E(z')}\mathrm dz'
\end{equation}
in which $E(z) \equiv \sqrt{\Omega_M (1+z)^3 + \Omega_\Lambda + \Omega_k (1+z)^2}$    .

\item
Transverse comoving distance
\begin{equation}
D_M = \frac{D_H}{\sqrt{\left\vert\Omega_k\right\vert}} f(\sqrt{\left\vert \Omega_k \right\vert}\frac {D_C}{D_H})
\end{equation}
in which 
\begin{equation}
f(x) = \Big\{
\begin{array}{lll}
\sinh(x), & \Omega_k>0 \quad &\text{hyperbola}\\
x, & \Omega_k = 0 &\text{flat}\\
\sin(x), & \Omega_k<0 &\text{parabola}
\end{array} .
\end{equation}


\item
Angular diameter distance
\begin{equation}
	D_A = \frac {1}{1+z} D_M
\end{equation}

We just divide the transverse distances by $1+z$
The advantage of it is that it is not singular at $z\rightarrow\infty$

\item
Luminosity distance
\begin{equation}
	D_L\equiv \sqrt{\frac{L}{4\pi S}}
\end{equation}
in which $L$ is bolometric luminosity\footnote{Bolometric luminosity: The total energy radiated by an object at all wavelengths, usually given in joules per second.}, $S$ is the bolometric flux.

Related to other distances
\begin{equation}
	D_L = (1+z)D_M = (1+z)^2 D_A
\end{equation}

\item
Distance modules
\begin{equation}
	DM \equiv 5 \log{D_L/10\mathrm pc}
\end{equation}

In Mpc unit,
\begin{equation}
	DM \equiv 5 \log{D_L} + 25
\end{equation}

\item
Comoving volume
\begin{equation}
	\mathrm dV_C = D_H \frac{(1+z)^2D_\Lambda^2}{E(z)}\mathrm d\Omega \mathrm dz
\end{equation}

\item
Lookback time
\begin{equation}
	t_L = t_H \int_0^z \frac{\mathrm dz'}{(1+z')E(z')}
\end{equation}
Lookback time is "the difference between the age $t_o$ of the Universe now (at observation) and the age $t_e$ of the Universe at the time the photons were emmitted (according to the object). "

\item
Probability of intersecting objects
\begin{equation}
	\mathrm dP = n(z)\sigma(z)D_H\frac{(1+z)^2}{E(z)}\mathrm dz
\end{equation}

"Given a population of objects with comoving number density $n(z)$ (number per unit volume) and cross section $\sigma(z)$ (area), what is the incremental probability $\mathrm dP$ atha a line of sight will intersect one of the objects in redshift interval $\mathrm dz$ at redshift $z$?"\footnote{{\bf arXiv:astro-ph/9905116v4}}


\end{enumerate}

This part is mostly taken from {\bf arXiv:astro-ph/9905116v4}

Page 418 of Gravitation And Cosmology: Principles And Applications Of The General Theory Of Relativity written by Weinberg in 1972.



















\section{The Homogeneous and Isotropic Universe}

It is always pointed out that most thoeries are based on the Cosmological Principle.

I have stated this principle of in the chapter telling the perturbation theory in cosmology.

\begin{quotation}
Empedocles: `God is an infinite sphere whose center is everywhere and circumference nowhere.'

Cosmological principle states that VIEWED on a sufficiently large scale, the properties of the Universe are the same for all observers. The key to understand this is that "same" means same physics principles and physical constants. 
Wikipedia gives three qualifications and two testable consequences. The two consequences are isotropic and homogeneous. Isotropic indicates no matter where we are looking at the universe is the same while homogeneous indicates no matter where we are located the universe is about the same (i.e., we are looking at a fair sample of the universe).

Problem is how to describe isotropy and homogeneity. [Liang, P360]
\end{quotation}

Actually, we have a rigorous form for the principle, as I mentioned in these paragraphs.

\paragraph{Homogeneous} 
Space-time $(M,g_{ab})$, sliced into ${\Sigma_t}$, induced metric of $g_{ab}$ on $\Sigma_t$. $(M, g_{ab})$ is spatially homogeneous if we can always find ${\Sigma_t}$, ensuring the existance of a set of isometry on $\Sigma_t$ itself.








































\section{Quantities}

\subsection{Energy-momentum Tensor}

Energy-momentum tensor can be made clear using a standard procedure of field theory.

Check out the table below. [From Ohanian and Ruffini's GRAVATATION AND SPACETIME, P494.]

\begin{center}
\begin{tabular}{ccc}
\hline\hline
                 & Particle & Field   \\
\hline
\vspace{1ex}Quantities & $q_i(t)$    &   $\phi(\vec x,t)$ \\
\vspace{1ex}independent quantitie & $t$,i[?]    &  $\vec x,t$ \\
\vspace{2ex}Lagrange    & $L=L(q_i,\dot{q_i})$  & $L=\int \mathscr L(\phi,\partial \phi/\partial x^\mu)\mathrm d^3x$ \\
\vspace{2ex}
Equation of motion & $\frac{\mathrm d}{\mathrm d t}\frac{\partial L}{\partial \dot{q_i}}-\frac{\partial L}{\partial q_i}=0$ & $\frac{\partial}{\partial x^\mu}\frac{\partial\mathscr L}{\partial \phi_{,\mu}}-\frac{\partial \mathscr L}{\partial \phi}=0$\\
\vspace{2ex}
Hamiltonian & $H=\sum \dot{q_i}\frac{\partial L}{\partial \dot q}-L$  & $H=\int (\phi_{,0}\frac{\partial \mathscr L}{\partial \phi_{,0}}-\mathscr L)\mathrm d^3x$\\
\hline\hline
\end{tabular}
\end{center}




The integrand in $H$ is the energy density of the complete system. We would like to choose the $t_0^{\phantom00}$ component of a canonical energy-momentum tensor energy-momentum tensor to describe energy density,
\begin{equation}
t_0^{\phantom00}=\phi_{,o}\frac{\partial\mathscr L}{\partial \phi_{,0}}-\mathscr L
.\end{equation}

It can be proved that the following equation is the only one that satisfies the constrain we proposed before.
\begin{equation}
t_\mu^{\phantom\mu\nu}=\phi_{,\mu}\frac{\partial \mathscr L}{\partial \phi_{,\nu}}-\delta_\mu^\nu\mathscr L
\end{equation}

We have the conservation law: 
\begin{equation}
\frac{\partial}{\partial x^\nu}t_\mu^{\phantom\mu\nu}=0
\end{equation}

Start from this we can find the Hamiltonian is conserved.
\begin{equation}
\frac{\mathrm dH}{\mathrm d t}=0
\end{equation}

The total momentum is conserved too.
\begin{equation}
P_k=\int t_k^{\phantom k0}\mathrm d^3x
\end{equation}

This is all about a scalar field. If we switch to vector field (EM field) and tensor field (1-order gravitation field), the canonical energy-momentum tensors are
\begin{eqnarray}
{t_{(em)}}_{\mu}^{\phantom\mu\nu}&=&A_{,\mu}^\alpha\frac{\partial \mathscr L_{(em)}}{\partial A_{,\nu}^\alpha}-\delta_\mu^\nu\mathscr L_{(em)} \\
{t_{(g1)}}_\mu^{\phantom\mu\nu}&=&h_{,\mu}^{\alpha\beta}\frac{\partial \mathscr L_{(g1)}}{\partial h_{,\nu}^{\alpha\beta}}-\delta_\mu^\nu\mathscr L_{(g1)}
\end{eqnarray}







In special relativity, the energy-momentum tensor for ideal fluid is 
\begin{eqnarray}
T_{ab}=(\rho+p)U_aU_b+p\eta_{ab}
\end{eqnarray}

Each component has its physical meaning.
\begin{enumerate}
\item
$T^{00}$ is the energy density; 
\item
$T^{0k}=T^{k0}$ is the $k$ momentum density (energy flux density);
\item
$T^{kl}=T^{lk}$ is the $k$ momentum flux density in $l$ direction.
\end{enumerate}

Just change all the $\eta_{ab}$ into $g_{ab}$.

For any system,
\begin{eqnarray}
\nabla_\mu T^{\mu\nu}&=&U^\mu U^\nu\nabla_\mu \rho+U^\mu U^\nu\nabla_\mu p+(\rho+p)U^\mu\nabla_\mu U^\nu+(\rho+p)U^\nu\nabla_\mu U^\mu+g^{\mu\nu}\nabla_\mu p  \\
&=&U^\mu U^\nu \dot\rho+(g^{\mu\nu}+U^\mu U^\nu)\nabla_\mu p+(\rho+p)A^\nu+(\rho+p)U^\nu\nabla_\mu \Theta \\
&=&U^\mu U^\nu\dot\rho+(\rho+p)U^\nu\nabla_\mu \Theta \\
&=&Q^\nu
\end{eqnarray}

$A^\nu$ vanishes because we have chosen a comoving observer. $\nabla_\mu p$ vanishes because in our standard model $\rho,p$ are coordinate free.
Actually, since $U^\mu$ is timelike, we have $(g^{\mu\nu}+U^\mu U^\nu)\nabla_\mu p=h^{\mu\nu}\nabla_\mu p$.






\subsection{Friedmann Universe}

\paragraph{Cosmological Principle}
Empedocles: `God is an infinite sphere whose center is everywhere and circumference nowhere.'

Cosmological principle states that VIEWED on a sufficiently large scale, the properties of the Universe are the same for all observers. The key to understand this is that "same" means same physics principles and physical constants. 
Wikipedia gives three qualifications and two testable consequences. The two consequences are isotropic and homogeneous. Isotropic indicates no matter where we are looking at the universe is the same while homogeneous indicates no matter where we are located the universe is about the same (i.e., we are looking at a fair sample of the universe).

Problem is how to describe isotropy and homogeneity. [Liang, P360]

\paragraph{Robertson-Walker Metric}
The next problem is how to set up a simple and useful coordinate system which metric to use in cosmology. [Liang, P366]

\paragraph{Evolution of Scale Factor}
Friedmann equation
\begin{eqnarray}
3(\dot a^2+k)/a^2&=&8\pi p \label{Friedmann_eqn} \\
2\ddot a/a+(\dot a^2+k)/a^2&=&-8\pi p  \label{ij-components}
\end{eqnarray}

\ref{Friedmann_eqn} is called Friedmann equation.

It is better to use another set of equations which are identical to \ref{Friedmann_eqn} and \ref{ij-components}
\begin{eqnarray}
\ddot a&=&-4\pi a(\rho+3p)/3 \\
0&=&\dot\rho+3(\rho+p)\dot a/a
\end{eqnarray}

These equations can be solved in some circumstances. [Liang, P376]

It could be useful to reform these equation.

\begin{eqnarray}
H^2+\frac{k}{a^2}&=&\frac{8}{3}\pi \rho    \\
2\dot H + 3H^2 +\frac{k}{a^2}&=&-8\pi p
\end{eqnarray}

Mixing
\begin{eqnarray}
3\ddot a&=& -4\pi a(\rho +3p)  \\
0 &=& \dot \rho +3 (\rho +p )\frac{\dot a}{a}
\end{eqnarray}


\begin{equation}
\dot H = -4\pi G (\rho + p) + \frac{k}{a^2}
\end{equation}


\backmatter







\end{document}
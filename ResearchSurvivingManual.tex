%This document had would had been used on Ways to Singularity, which is a website that supports MathJax. So some html elements may occur in this document. DELETE THIS when publishing. (If you are not very clear on the grammar I used here, read the academic publication called Time Traveller's Handbook of 1001 Tense Formations by Dr Dan Streetmentioner, which had would had been publish in year 220010 of Gregorian Calendar.)
\documentclass[11pt]{book}
\usepackage{amsmath,amsthm,amsfonts,amssymb,bm}

%\allowdisplaybreaks
\usepackage{indentfirst}


\usepackage{enumerate}

%\usepackage{makeidx}
\usepackage{mathrsfs}
\usepackage{color}
%\usepackage{ulem}  %\sout{stroke one text}
%\usepackage{hyperref}                 % For creating hyperlinks in cross references


%\includeonly{}


\begin{document}

\title{Research Survival Handbook \\ (\textbf{Unfinished})}
\author{{\bf MA} Lei  \\
@ Interplanetary Immigration Agency \\
{\small\em \copyright \ Draft date \today}}
\date{}
%\begin{document}
\maketitle

\frontmatter
\tableofcontents


\chapter{Preface}

I have a bad memory that I can hardly remember anything.

To get rid of it, I tried many ways of help myself memorizing things and pushing myself to the frontier of physics. Finally, I decided to collect some of my notes together and established this project on github.\url{http://cosmologytaskforce.github.com/CosmologyResearchSurvivingManual}

This is only a draft handbook for myself in principal. However, I believe everyone need a handbook of his/her area and my version of handbook might be helpful for some people working on similiar things with mine.

This book can never not be delivered formally because I borrowed many resources in this book. Here I list the most important of them here.

\begin{itemize}
\item
In the part of physics, I just take the source file of a note on GRE Physics subject test written by Lin Cong and typeset by Duncan Watts and rearranged the sections.\url{http://www.hcs.harvard.edu/~physics/?q=node/13};\url{http://www.hcs.harvard.edu/~physics/files/GRE\%20notes.tex} Then I did modifications based on it.
He said in his notes everyone is welcome to typeset and improve his notes but if you are going to use this document commercially you need to contact him first.
\end{itemize}





\iffalse
\chapter{Preface}

I have a bad memory, very bad. So bad that I can hardly remember anything.

I tried many many ways of pushing myself to the frontier of physics. That bad memory really pisses me off. So I decided to borrow the power of paper and computers.
\fi



\mainmatter
\part{Fundamental Physics}

\chapter{Basic}

\section{Dimension}

How to find the relationship between two quantities? For example, what is the dimensional relationship between length and mass.
\begin{quotation}
Plank constant: $\hbar \sim [Energy]\cdot [Time] \sim [Mass]\cdot [Length]^2 \cdot [Time]^{-1}$ 

Speed of light in vacuum: $c\sim [Length]\cdot [Time]^{-1}$

Gravitational constant: $G \sim [Length]^3\cdot [Mass]^{-1} \cdot [Time]^{-2}$
\end{quotation}

Then it is easy to find that a combination of $c/\hbar$ cancels the dimension of mass and leaves the inverse of length. That is
\begin{equation}
[Length]^2 = \frac{\hbar G}{c^3}
\end{equation}



\section{Equations That Should Never Be Forgotten}

\subsection{Electrodynamics}

\subsubsection{Maxwell Equations}
\begin{eqnarray}
\nabla\times\vec E&=&-\partial_t \vec B \\
\nabla\times\vec H&=&\vec J+\partial_t \vec D \\
\nabla\cdot \vec D&=&\rho \\
\nabla\cdot \vec B&=&0
\end{eqnarray}

For linear meterials, \begin{eqnarray}
\vec D&=&\epsilon \vec E \\
\vec B&=&\mu \vec H \\
\vec J&=& \sigma \vec E
\end{eqnarray}


Hamilton conanical equations

\begin{eqnarray}
\dot q_i &=& \frac{\partial H}{\partial p_i}  \\
\dot p_i &=& - \frac{\partial H}{\partial q_i}
\end{eqnarray}


Liouville's Law
\begin{eqnarray}
\frac{\mathrm d \rho}{\mathrm d t}\equiv \frac{\partial \rho}{\partial t} + \sum_i \left[ \frac{\partial \rho}{\partial q_i}\dot q_i + \frac{\partial \rho}{\partial p_i}\dot p_i \right] = 0
\end{eqnarray}














\part{Advanced Physics}

\chapter{Relativy}
\section{How to survive the calculations of Special Relativity}

\subsection{Important Relations}
\begin{quotation}
Metric in use
\begin{equation}\eta_{\mu\nu}=\left(\begin{matrix}
	-1 & 0 & 0 & 0\\
	0 & 1 & 0 & 0\\
	0 & 0 & 1 & 0\\
	0 & 0 & 0 & 1\\
\end{matrix}\right)\end{equation}
\end{quotation}




\section{Quantities and Operations}

\paragraph{d'Alembertian}
d'Alembert operator, or wave operator, is the Lapace operator in Minkowski space.
\footnote{Actually, there are more general definations for Lapacian, which includes this d'Alembertian of course.}
\begin{eqnarray}
\Box\equiv \partial_\mu\partial^\nu&=&\eta_{\mu\nu}\partial^\mu \partial^\nu
\end{eqnarray}

In the usual {t,x,y,z} natural orthonormal basis,
\begin{eqnarray}
 \Box&=&-\partial_t^2+\partial_x^2+\partial_y^2+\partial_z^2 \\
&=&-\partial_t^2+\Delta^2 \\
&=&-\partial_t^2+\nabla
\end{eqnarray}

\begin{quotation}
On wiki \footnote{wiki:D'Alembert\_operator}, they give some applications to it.
\paragraph{klein-Gordon equation} $(\Box+m^2)\phi=0$
\paragraph{wave equation for electromagnetic field in vacuum} For the electromagnetic four-potential $\Box A^\mu=0$\footnote{Gauge}
\paragraph{wave equation for small vibrations} $\Box_c u(t,x)=0\rightarrow u_{tt}-c^2 u_{xx}=0$
\end{quotation}




\section{Fields and Particles}

\subsection{Energy-Momentum Tensor for Particles}

\begin{equation}
S_p \equiv -m c \int \int \mathrm d s\mathrm d\tau \sqrt{-\dot x ^\mu g_{\mu\nu} \dot x^\nu} \delta^4(x^\mu - x^\mu (s))    ,
\end{equation}
in which $x^\mu(s)$ is the trajectory of the particle. Then the energy density $\rho$ corresponds to $m\delta^4(x^\mu- x^\mu(s))$.

The Largrange density
\begin{equation}
\mathcal L = -\int\mathrm ds mc \sqrt{-\dot x^\mu g_{\mu\nu}\dot x^\nu}\delta^4(x^\mu - x^\mu(s))
\end{equation}

Energy-momentum density is $\mathcal T^{\mu\nu} = \sqrt{-g}T^{\mu\nu}$ is
\begin{equation} 
\mathcal T^{\mu\nu} = -2 \frac{\partial \mathcal L}{\partial g_{\mu\nu}}
\end{equation}

Finally,
\begin{eqnarray}
\mathcal T^{\mu\nu} &=& \int \mathrm ds \frac{mc\dot x^\mu \dot x^\nu}{\sqrt{-\dot x^\mu g_{\mu\nu} \dot x^\nu}} \delta(t-t(s))\delta^3(\vec x - \vec x(t)) \\
&=& m\dot x^\mu \dot x^\nu \frac{\mathrm d s}{\mathrm d t} \delta^3(\vec x - \vec x(s(t)))
\end{eqnarray}






\section{Theorems}

\subsection{Killing Vector Related}

\newtheorem{theorem}{}[chapter]

\begin{theorem}[]
$\xi^a$ is Killing vector field, $T^a$ is the tangent vector of geodesic line. Then $T^a\nabla_a(T^b\xi_b)=0$, that is $T^b\xi_b$ is a constant on geodesics.
\end{theorem}




\section{Topics}
\subsection{Redshift}

In geometrical optics limit, the angular frequency $\omega$ of a photon with a 4-vector $K^a$, measured by a observer with a 4-velocity $Z^a$, is $\omega=-K_aZ^a$.









\part{Tools}

\chapter{Mathematics}

\section{Differential Geometry}

\subsection{Metric}

\subsubsection{Definations}

Denote the basis in use as $\hat e_\mu$, then the metric can be written as
\begin{equation}
g_{\mu\nu}=\hat e_\mu \hat \cdot e_\nu
\end{equation}
if the basis satisfies

Inversed metric
\begin{equation}
g_{\mu\lambda}g^{\lambda\nu}=\delta_\mu^\nu = g_\mu^\nu
\end{equation}





\subsubsection{How to calculate the metric}

Let's check the definition of metric again.

If we choose a basis $\hat e_\mu$, then a vector (at one certain point) in this coordinate system is
\begin{equation}
x^a=x^\mu \hat e_\mu
\end{equation}

Then we can construct the expression of metric of this point under this coordinate system,
\begin{equation}
g_{\mu\nu}=\hat e_\mu\cdot \hat e_\nu
\end{equation}

For example, in spherical coordinate system, 
\begin{equation}
\vec x=r\sin \theta\cos\phi \hat e_x+r\sin\theta\sin\phi \hat e_y+r\cos\theta \hat e_z \label{eq:relativity_metric_point}
\end{equation}



Now we have to find the basis under spherical coordinate system. Assume the basis is $\hat e_r, \hat e_\theta, \hat e_\phi$. Choose some scale factors $h_r=1, h_\theta=r, h_\phi=r\sin\theta$. Then the basis is
$\hat e_r=\frac{\partial \vec x}{h_r\partial r}=\hat e_x \sin\theta\cos\phi+\hat e_y \sin\theta\sin\phi+\hat e_z \cos\theta$, etc. Then collect the terms in formula \ref{eq:relativity_metric_point} is we get $\vec x=r\hat e_r$, this is incomplete. So we check the derivative.
\begin{eqnarray}
\mathrm d\vec x&=& \hat e_x (\mathrm dr \sin\theta\cos\phi+r\cos\theta\cos\phi\mathrm d\theta-r\sin\theta\sin\phi\mathrm d\phi)\\
&&\hat e_y (\mathrm dr\sin\theta\sin\phi+r\cos\theta\sin\phi\mathrm d\theta+r\sin\theta\cos\phi\mathrm d\phi) \\
&&\hat e_z (\mathrm dr\cos\theta-r\sin\theta\mathrm d\theta) \\
&=&\mathrm dr(\hat e_x\sin\theta\cos\phi +\hat e_y \sin\theta\sin\phi -\hat e_z \cos\theta)  \\
&&\mathrm d\theta (\hat e_x\cos\theta\cos\phi +\hat e_y \cos\theta\sin\phi - \hat e_z \sin\theta)r \\
&&\mathrm d\phi (-\hat e_x\sin\phi +\hat e_y \cos\phi)r\sin\theta  \\
&=&\hat e_r\mathrm dr+\hat e_\theta r\mathrm d\theta +\hat e_\phi r\sin\theta\mathrm d \phi
\end{eqnarray}

Once we reach here, the component ($e_r ,e_\theta, e_\phi$) of the point under the spherical coordinates system basis ($\hat e_r, \hat e_\theta, \hat e_\phi$) at this point are clear, i.e.,

\begin{eqnarray}
\mathrm d\vec x&=&\hat e_r\mathrm d r+\hat e_\theta r\mathrm d \theta+\hat e_\phi r\sin\theta \mathrm d\phi \\
&=&e_r\mathrm d r+e_\theta \mathrm d\theta+e_\phi \mathrm d\phi
\end{eqnarray}

In this way, the metric tensor for spherical coordinates is 
\begin{equation}
g_{\mu\nu}=(e_\mu\cdot e_\nu)=\left(\begin{matrix}
1 &0&0 \\
0& r^2&0 \\
0&0& r^2\sin^2\theta \\
\end{matrix}\right)
\end{equation}



\subsection{Connection}

First class connection can be calcuated 
\begin{equation}
\Gamma^\mu_{\phantom{\mu}\nu\lambda}=\hat e^\mu\cdot \hat e_{\mu,\lambda}
\end{equation}

Second class connection is\footnote{Kevin E. Cahill}
\begin{equation}
[\mu\nu,\iota]=g_{\iota\mu}\Gamma^\mu_{\phantom{\mu}\nu\lambda}
\end{equation}




\subsection{Gradient, Curl, Divergence, etc}

\paragraph{Gradient} 
\begin{equation}
T^b_{\phantom bc;a}= \nabla_aT^b_{\phantom bc}=T^b_{\phantom bc,a}+\Gamma^b_{ad}T^d_{\phantom dc}-\Gamma^d_{ac}T^b_{\phantom bd}
\end{equation}

\paragraph{Curl}For an anti-symmetric tensor, $a_{\mu\nu}=-a_{\nu\mu}$
\begin{eqnarray}
\mathrm{Curl}_{\mu\nu\tau}(a_{\mu\nu})&\equiv& a_{\mu\nu;\tau}+a_{\nu\tau;\mu}+a_{\tau\mu;\nu} \\
&=&a_{\mu\nu,\tau}+a_{\nu\tau,\mu}+a_{\tau\mu,\nu}
\end{eqnarray}

\paragraph{Divergence}

\begin{eqnarray}
\mathrm{div}_\nu(a^{\mu\nu})&\equiv& a^{\mu\nu}_{\phantom{\mu\nu};\nu}=\frac{\partial a^{\mu\nu}}{\partial x^\nu}+\Gamma^\mu_{\nu\tau}a^{\tau\nu}+\Gamma^\nu_{\nu\tau}a^{\mu\tau} \\
&=&\frac1{\sqrt{-g}}\frac{\partial}{\partial x^\nu}(\sqrt{-g}a^{\mu\nu})+\Gamma^\mu_{\nu\lambda}a^{\nu\lambda}
\end{eqnarray}

For an anti-symmetric tensor
\begin{equation}
\mathrm {div}(a^{\mu\nu})=\frac1{\sqrt{-g}}\frac{\partial}{\partial x^\nu}(\sqrt{-g}a^{\mu\nu})
\end{equation}

\subparagraph{Annotation} Using the relation $g=g_{\mu\nu}A_{\mu\nu}$, $A_{\mu\nu}$ is the algebraic complement, we can prove the following two equalities.
\begin{equation}
\Gamma^\mu_{\mu\nu}=\partial_\nu\ln{\sqrt{-g}}
\end{equation}

\begin{equation}
V^\mu_{\phantom\mu;\mu}=\frac1{\sqrt{-g}}\frac{\partial}{\partial x^\mu}(\sqrt{-g}V^\mu)
\end{equation}

In some simple case, all the three kind of operation can be demonstrated by different applications of the del operator, which $\nabla\equiv \hat x\partial_x+\hat y\partial_y+\hat z \partial_z$. \\
Gradient,  $\nabla f$, in which $f$ is a scalar. \\
Divergence, $\nabla\cdot \vec v$ \\
Curl, $\nabla \times \vec v$
Laplacian, $\Delta\equiv \nabla\cdot\nabla\equiv \nabla^2$


\section{Linear Algebra}

\subsection{Basic Concepts}

\paragraph{Trace}
Trace should be calculated using the metrc. An example is the trace of Ricci tensor,
\begin{equation}
R=g^{ab}R_{ab}
\end{equation}

Einstein equation is \begin{equation}
R_{ab}-\frac{1}{2}g_{ab}R=8\pi G T_{ab}
\end{equation}
 The trace is \begin{eqnarray}
g^{ab}R_{ab}-\frac{1}{2}g^{ab}g_{ab}R&=&8\pi G g^{ab}T_{ab} \\
\Rightarrow R-\frac{1}{2} 4 R &=& 8\pi G T \\
\Rightarrow -R&=&8\pi GT
\end{eqnarray}






\section{Differential Equations}

\subsection{Standard Procedure}


\subsection{Tricky}

\paragraph{WKB Approximation}

When the highest derivative is multiplied by a small parameter, try this.









\chapter{Cosmology}


\section{What's in the begining}

In cosmology, the frame used most frequently is the cosmic microwave background radiation (CMB) stationary frame,  not earth frame. We sometimes say it is earth frame because the speed of earth relative to CMB is rather small compared to the galaxies' movement observed by we earth beings.



\section{Constants And Physical Quantities}

\begin{itemize}
\item
Deceleration parameter of today, 
\[q=-\left( \frac{a}{\dot a}\ddot a \right)=-\frac{\ddot a}{\dot a}\frac{1}{H(a)}\]
Deceleration parameter is tightly related to Friedmann equation.
\footnote{Firedmann equation can be written in terms of deceleration parameter,\[q=1/2(1+3w)(1+k/(aH)^2)\]}


\end{itemize}


\subsection{Cosmographic Parameters}
\begin{enumerate}[\bf\tiny{Co-Graphy-Par}-1]
\item
Recession speed
\begin{equation}
v = H_0 \cdot d
\end{equation}

\item
Hubble time
\begin{equation}
	t_H = \frac 1 H_0
\end{equation}

\item
Hubble distance
\begin{equation}
	D_H = \frac c H_0 = c \cdot t_H
\end{equation}

\item
Dimensionless density parameters

Matter
\begin{equation}
	\Omega_M = \frac{8\pi G \rho_M |_{a=1}}{3H_0^2}
\end{equation}

Lambda
\begin{equation}
	\Omega_\Lambda = \frac{\Lambda c^2}{3H_0^2}
\end{equation}

Curvature can be viewed as some energy, too.
From Friedmann equation we can sort out it. Or equivalently
\begin{equation}
	\Omega_k = 1- \Omega_M- \Omega_\Lambda
\end{equation}

\end{enumerate}




\subsubsection{Redshifts and Distances}

\begin{enumerate}[\bf\tiny{Co-Dis}-1]
\item
Redshift defined in observation
\begin{equation}
	z = \frac{\nu_e}{\nu_o} -1 = \frac{\lambda_o-\lambda_e}{\lambda_e}
\end{equation}

\item
In linear range, given a radial velocity, redshift can be written as
\begin{equation}
	z = \sqrt{\frac{1 + v/c}{1 - v/c}} - 1
\end{equation}

\item
Peculiar velocity
\begin{equation}
	v_{pec} = c \frac {z_{obs} - z_{cos} }{ 1+z }  ,
\end{equation}
in which,$z_{obs}$ is the observed redshift, while $z_{cos}$ is the cosmological redshift. Cosmological redshift is the Hubble flow "due solely to the expansion of the universe".
{\bf This definition is valid when $v_{pec}$ is much smaller than speed of light, i.e., $v_{pec}\ll c$.}

\item
Scale factor
\begin{equation}
	a|_{t_0} = (1+z) a|_{t_e}  \label{eq-co-dis-scale_factor}  .
\end{equation}
In cosmology, we often use another form which is derived from \label{eq-co-dis-scale_factor} and setting $a|_{t_0}= 1$
\begin{equation}
	a = \frac 1{1+z}     .
\end{equation}
This is the scale factor we used in line element.

\item
Relative redshift
\begin{equation}
	z_{12} = \frac{a|_{t_1}}{a|_{t_2}}-1 = \frac{1+z_2}{1+z_1}
\end{equation}

\item
Line-of-sight comoving distance
\begin{equation}
	D_c = D_H\int^z_0 \frac{1}{E(z')}\mathrm dz'
\end{equation}
in which $E(z) \equiv \sqrt{\Omega_M (1+z)^3} + \Omega_\Lambda + \Omega_k (1+z)^2$    .

\item
Transverse comoving distance
\begin{equation}
D_M = \frac{D_H}{\sqrt{\left\vert\Omega_k\right\vert}} f(\sqrt{\left\vert \Omega_k \right\vert}\frac {D_C}{D_H})
\end{equation}
in which 
\begin{equation}
f(x) = \Big\{
\begin{array}{lll}
\sinh(x), & \Omega_k>0 \quad &\text{hyperbola}\\
x, & \Omega_k = 0 &\text{flat}\\
\sin(x), & \Omega_k<0 &\text{parabola}
\end{array} .
\end{equation}


\item
Angular diameter distance
\begin{equation}
	D_A = \frac {1}{1+z} D_M
\end{equation}

We just divide the transverse distances by $1+z$
The advantage of it is that it is not singular at $z\rightarrow\infty$

\item
Luminosity distance
\begin{equation}
	D_L\equiv \sqrt{\frac{L}{4\pi S}}
\end{equation}
in which $L$ is bolometric luminosity\footnote{Bolometric luminosity: The total energy radiated by an object at all wavelengths, usually given in joules per second.}, $S$ is the bolometric flux.

Related to other distances
\begin{equation}
	D_L = (1+z)D_M = (1+z)^2 D_A
\end{equation}

\item
Distance modules
\begin{equation}
	DM \equiv 5 \log{D_L/10\mathrm pc}
\end{equation}

In Mpc unit,
\begin{equation}
	DM \equiv 5 \log{D_L} + 25
\end{equation}

\item
Comoving volume
\begin{equation}
	\mathrm dV_C = D_H \frac{(1+z)^2D_\Lambda^2}{E(z)}\mathrm d\Omega \mathrm dz
\end{equation}

\item
Lookback time
\begin{equation}
	t_L = t_H \int_0^z \frac{\mathrm dz'}{(1+z')E(z')}
\end{equation}
Lookback time is "the difference between the age $t_o$ of the Universe now (at observation) and the age $t_e$ of the Universe at the time the photons were emmitted (according to the object). "

\item
Probability of intersecting objects
\begin{equation}
	\mathrm dP = n(z)\sigma(z)D_H\frac{(1+z)^2}{E(z)}\mathrm dz
\end{equation}

"Given a population of objects with comoving number density $n(z)$ (number per unit volume) and cross section $\sigma(z)$ (area), what is the incremental probability $\mathrm dP$ atha a line of sight will intersect one of the objects in redshift interval $\mathrm dz$ at redshift $z$?"\footnote{{\bf arXiv:astro-ph/9905116v4}}


\end{enumerate}

This part is mostly taken from {\bf arXiv:astro-ph/9905116v4}

Page 418 of Gravitation And Cosmology: Principles And Applications Of The General Theory Of Relativity written by Weinberg in 1972.



















\section{The Homogeneous and Isotropic Universe}

It is always pointed out that most thoeries are based on the Cosmological Principle.

I have stated this principle of in the chapter telling the perturbation theory in cosmology.

\begin{quotation}
Empedocles: `God is an infinite sphere whose center is everywhere and circumference nowhere.'

Cosmological principle states that VIEWED on a sufficiently large scale, the properties of the Universe are the same for all observers. The key to understand this is that "same" means same physics principles and physical constants. 
Wikipedia gives three qualifications and two testable consequences. The two consequences are isotropic and homogeneous. Isotropic indicates no matter where we are looking at the universe is the same while homogeneous indicates no matter where we are located the universe is about the same (i.e., we are looking at a fair sample of the universe).

Problem is how to describe isotropy and homogeneity. [Liang, P360]
\end{quotation}

Actually, we have a rigorous form for the principle, as I mentioned in these paragraphs.

\paragraph{Homogeneous} 
Space-time $(M,g_{ab})$, sliced into ${\Sigma_t}$, induced metric of $g_{ab}$ on $\Sigma_t$. $(M, g_{ab})$ is spatially homogeneous if we can always find ${\Sigma_t}$, ensuring the existance of a set of isometry on $\Sigma_t$ itself.








































\section{Quantities}

\subsection{Energy-momentum Tensor}

Energy-momentum tensor can be made clear using a standard procedure of field theory.

Check out the table below. [From Ohanian and Ruffini's GRAVATATION AND SPACETIME, P494.]

\begin{center}
\begin{tabular}{ccc}
\hline\hline
                 & Particle & Field   \\
\hline
\vspace{1ex}Quantities & $q_i(t)$    &   $\phi(\vec x,t)$ \\
\vspace{1ex}independent quantitie & $t$,i[?]    &  $\vec x,t$ \\
\vspace{2ex}Lagrange    & $L=L(q_i,\dot{q_i})$  & $L=\int \mathscr L(\phi,\partial \phi/\partial x^\mu)\mathrm d^3x$ \\
\vspace{2ex}
Equation of motion & $\frac{\mathrm d}{\mathrm d t}\frac{\partial L}{\partial \dot{q_i}}-\frac{\partial L}{\partial q_i}=0$ & $\frac{\partial}{\partial x^\mu}\frac{\partial\mathscr L}{\partial \phi_{,\mu}}-\frac{\partial \mathscr L}{\partial \phi}=0$\\
\vspace{2ex}
Hamiltonian & $H=\sum \dot{q_i}\frac{\partial L}{\partial \dot q}-L$  & $H=\int (\phi_{,0}\frac{\partial \mathscr L}{\partial \phi_{,0}}-\mathscr L)\mathrm d^3x$\\
\hline\hline
\end{tabular}
\end{center}




The integrand in $H$ is the energy density of the complete system. We would like to choose the $t_0^{\phantom00}$ component of a canonical energy-momentum tensor energy-momentum tensor to describe energy density,
\begin{equation}
t_0^{\phantom00}=\phi_{,o}\frac{\partial\mathscr L}{\partial \phi_{,0}}-\mathscr L
.\end{equation}

It can be proved that the following equation is the only one that satisfies the constrain we proposed before.
\begin{equation}
t_\mu^{\phantom\mu\nu}=\phi_{,\mu}\frac{\partial \mathscr L}{\partial \phi_{,\nu}}-\delta_\mu^\nu\mathscr L
\end{equation}

We have the conservation law: 
\begin{equation}
\frac{\partial}{\partial x^\nu}t_\mu^{\phantom\mu\nu}=0
\end{equation}

Start from this we can find the Hamiltonian is conserved.
\begin{equation}
\frac{\mathrm dH}{\mathrm d t}=0
\end{equation}

The total momentum is conserved too.
\begin{equation}
P_k=\int t_k^{\phantom k0}\mathrm d^3x
\end{equation}

This is all about a scalar field. If we switch to vector field (EM field) and tensor field (1-order gravitation field), the canonical energy-momentum tensors are
\begin{eqnarray}
{t_{(em)}}_{\mu}^{\phantom\mu\nu}&=&A_{,\mu}^\alpha\frac{\partial \mathscr L_{(em)}}{\partial A_{,\nu}^\alpha}-\delta_\mu^\nu\mathscr L_{(em)} \\
{t_{(g1)}}_\mu^{\phantom\mu\nu}&=&h_{,\mu}^{\alpha\beta}\frac{\partial \mathscr L_{(g1)}}{\partial h_{,\nu}^{\alpha\beta}}-\delta_\mu^\nu\mathscr L_{(g1)}
\end{eqnarray}







In special relativity, the energy-momentum tensor for ideal fluid is 
\begin{eqnarray}
T_{ab}=(\rho+p)U_aU_b+p\eta_{ab}
\end{eqnarray}

Each component has its physical meaning.
\begin{enumerate}
\item
$T^{00}$ is the energy density; 
\item
$T^{0k}=T^{k0}$ is the $k$ momentum density (energy flux density);
\item
$T^{kl}=T^{lk}$ is the $k$ momentum flux density in $l$ direction.
\end{enumerate}

Just change all the $\eta_{ab}$ into $g_{ab}$.

For any system,
\begin{eqnarray}
\nabla_\mu T^{\mu\nu}&=&U^\mu U^\nu\nabla_\mu \rho+U^\mu U^\nu\nabla_\mu p+(\rho+p)U^\mu\nabla_\mu U^\nu+(\rho+p)U^\nu\nabla_\mu U^\mu+g^{\mu\nu}\nabla_\mu p  \\
&=&U^\mu U^\nu \dot\rho+(g^{\mu\nu}+U^\mu U^\nu)\nabla_\mu p+(\rho+p)A^\nu+(\rho+p)U^\nu\nabla_\mu \Theta \\
&=&U^\mu U^\nu\dot\rho+(\rho+p)U^\nu\nabla_\mu \Theta \\
&=&Q^\nu
\end{eqnarray}

$A^\nu$ vanishes because we have chosen a comoving observer. $\nabla_\mu p$ vanishes because in our standard model $\rho,p$ are coordinate free.
Actually, since $U^\mu$ is timelike, we have $(g^{\mu\nu}+U^\mu U^\nu)\nabla_\mu p=h^{\mu\nu}\nabla_\mu p$.






\subsection{Friedmann Universe}

\paragraph{Cosmological Principle}
Empedocles: `God is an infinite sphere whose center is everywhere and circumference nowhere.'

Cosmological principle states that VIEWED on a sufficiently large scale, the properties of the Universe are the same for all observers. The key to understand this is that "same" means same physics principles and physical constants. 
Wikipedia gives three qualifications and two testable consequences. The two consequences are isotropic and homogeneous. Isotropic indicates no matter where we are looking at the universe is the same while homogeneous indicates no matter where we are located the universe is about the same (i.e., we are looking at a fair sample of the universe).

Problem is how to describe isotropy and homogeneity. [Liang, P360]

\paragraph{Robertson-Walker Metric}
The next problem is how to set up a simple and useful coordinate system which metric to use in cosmology. [Liang, P366]

\paragraph{Evolution of Scale Factor}
Friedmann equation
\begin{eqnarray}
3(\dot a^2+k)/a^2&=&8\pi p \label{Friedmann_eqn} \\
2\ddot a/a+(\dot a^2+k)/a^2&=&-8\pi p  \label{ij-components}
\end{eqnarray}

\ref{Friedmann_eqn} is called Friedmann equation.

It is better to use another set of equations which are identical to \ref{Friedmann_eqn} and \ref{ij-components}
\begin{eqnarray}
\ddot a&=&-4\pi a(\rho+3p)/3 \\
0&=&\dot\rho+3(\rho+p)\dot a/a
\end{eqnarray}

These equations can be solved in some circumstances. [Liang, P376]

It could be useful to reform these equation.

\begin{eqnarray}
H^2+\frac{k}{a^2}&=&\frac{8}{3}\pi \rho    \\
2\dot H + 3H^2 +\frac{k}{a^2}&=&-8\pi p
\end{eqnarray}

Mixing
\begin{eqnarray}
3\ddot a&=& -4\pi a(\rho +3p)  \\
0 &=& \dot \rho +3 (\rho +p )\frac{\dot a}{a}
\end{eqnarray}


\begin{equation}
\dot H = -4\pi G (\rho + p) + \frac{k}{a^2}
\end{equation}







\backmatter





\end{document}